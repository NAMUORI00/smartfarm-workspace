\section{Related Work Comparison Tables}
\label{app:related-work}

본 부록에서는 스마트팜 RAG 시스템 관련 연구를 표 형식으로 정리한다.

\subsection{RAG 연구 발전 과정}

RAG(Retrieval-Augmented Generation)는 검색과 생성을 결합하여 LLM의 환각을 줄이는 접근법이다. 본 절에서는 RAG의 발전 과정을 시간순으로 정리하고, 그래프 기반 RAG와 농업 도메인 적용 연구를 포함한다.

\subsubsection{초기 RAG 연구 (2020--2022)}

\begin{table}[H]
\centering
\caption{초기 RAG 연구 (2020--2022)}
\label{tab:rag-early}
\small
\begin{tabular}{p{2cm}cp{5cm}p{5cm}}
\toprule
\textbf{연구} & \textbf{연도} & \textbf{주요 기여} & \textbf{한계점} \\
\midrule
RAG (Lewis et al.) [1] & 2020 & 검색+생성 결합 → LLM이 모르는 정보도 외부 문서에서 찾아 답변, 환각 감소 & 클라우드 GPU 필수 → 농가 현장 배포 불가 \\
DPR (Karpukhin) [2] & 2020 & Dense retrieval(BERT 기반 의미 검색) → 정확한 단어 없어도 비슷한 뜻의 문서 찾음, 정확도 +15\% & "EC 2.5 dS/m" 같은 수치 검색 실패 → 농업 수치 질의 부정확 \\
BEIR Benchmark [3] & 2021 & 18개 데이터셋 평가 → Hybrid(키워드+의미 결합)가 Dense만보다 우수 입증 & 농업 데이터셋 미포함 → 도메인 성능 검증 안됨 \\
\bottomrule
\end{tabular}
\end{table}

\subsubsection{개선된 RAG 기법 (2023--2024)}

\begin{table}[H]
\centering
\caption{개선된 RAG 기법 (2023--2024)}
\label{tab:rag-advanced}
\small
\begin{tabular}{p{2cm}cp{5cm}p{5cm}}
\toprule
\textbf{연구} & \textbf{연도} & \textbf{주요 기여} & \textbf{한계점} \\
\midrule
RAG Survey (Gao et al.) [4] & 2024 & RAG 발전 단계 분류 (Naive→Advanced→Modular) → 설계 시 참조할 체계적 가이드 제공 & 엣지/오프라인 미언급 → 네트워크 불안정 현장 적용 어려움 \\
Graph RAG Survey [5] & 2024 & 문서 간 관계를 그래프로 연결한 RAG 50편 분석 → 구조화된 지식 활용법 정리 & 농업 적용 사례 미포함 (2024년 기준) \\
GraphRAG (MS) [6] & 2024 & 문서들을 지식 그래프로 구조화 → "전체 문서 요약해줘" 같은 전역 질의(global query) 가능 & GPT-4 수천 회 호출 필요 → 문서 1000개당 \$100+ 비용 \\
LightRAG [7] & 2024 & GraphRAG 경량화 → 동일 품질에 비용 10배↓ & 여전히 LLM으로 NER(Named Entity Recognition) → 엣지 실행 불가 \\
Cluster-based Fusion [8] & 2024 & Sparse(키워드 매칭)로 후보 추린 뒤 Dense 적용 → 검색 속도 2배 향상 & 2가지 방식만 결합, 가중치 고정 → 질의 유형별 최적화 불가 \\
EdgeRAG [9] & 2024 & 계층적 인덱싱(hierarchical indexing) → 필요한 임베딩만 로드하여 메모리 제약 해결 & Dense 검색만 지원 → 수치 검색 약함, 도메인 지식 활용 안함 \\
\bottomrule
\end{tabular}
\end{table}

\subsubsection{최신 농업 도메인 RAG 연구 (2025)}

\begin{table}[H]
\centering
\caption{최신 농업 도메인 RAG 연구 (2025)}
\label{tab:rag-agriculture-2025}
\small
\begin{tabular}{p{2.5cm}cp{5cm}p{4.5cm}}
\toprule
\textbf{연구} & \textbf{연도} & \textbf{주요 기여} & \textbf{한계점} \\
\midrule
KG+LLM in Agriculture [10] & 2025 & 농업 지식그래프+LLM 결합 동향 최초 정리 → 농업 AI 연구 방향 제시 & 실제 배포 방안 미제시 → 연구-현장 격차 해소 안됨 \\
PathRAG (Chen) [11] & 2025 & 관계 경로(relational path) 기반 검색 → 비슷한 문서 중복 감소, 다양한 정보 제공 & 범용 관계 유형만 사용 → "원인→증상→해결책" 인과 추론 안됨 \\
Crop GraphRAG (Wu) [12] & 2025 & 병해충 KG+RAG 융합 → 환각 억제, 농업 도메인 QA 정확도 향상 & 엣지/오프라인 환경 미언급 \\
AHR-RAG (Yang) [13] & 2025 & 적응형 하이브리드 검색(단일홉/다중홉) → 91만 트리플릿 KB, 복잡 질의 대응 & 91만 트리플릿 규모 KB 전제 \\
ReG-RAG (Li) [14] & 2025 & 쿼리 재작성 + 지식그래프 강화 → 질의 의도 명확화, 검색 정확도 향상 & 클라우드 LLM 의존 → 엣지 배포 불가 \\
AgroMetLLM (Ray) [15] & 2025 & Raspberry Pi 4B에서 양자화 LLM → 오프라인 1-2초 응답 & 증발산(ET) 예측 특화 → 범용 Q\&A 불가 \\
Agri-LLM (Jiang) [16] & 2025 & 정밀 지식 검색 + 협업 생성 → 농업 LLM 품질 향상 & 엣지 배포 미고려 \\
\bottomrule
\end{tabular}
\end{table}

\textbf{본 연구 대응:}
\begin{itemize}
\item \textbf{HybridDAT 독자 설계} → Dense-Sparse-PathRAG 3채널 융합, 농업 도메인 특화 하이브리드 시스템
\item \textbf{농업 도메인 적응}: 스마트팜 특화 엔티티 타입 6종(crop, disease, environment, practice, nutrient, stage), 농업 온톨로지 기반 엔티티 추출 가이드
\item \textbf{엣지 환경 최적화}: Q4\_K\_M 양자화 → 8GB RAM 엣지 디바이스에서 전체 RAG 파이프라인 배포
\item 작물별 검색 필터링 → "와사비 질문에 상추 문서" 문제 해결
\item FAISS mmap 기반 인덱스 로딩 → 대용량 인덱스도 저메모리에서 사용 가능
\item 메모리 적응형 리랭커 선택 → 가용 RAM에 맞춰 최선의 품질 제공
\end{itemize}

\subsection{하이브리드 검색 기법}

Dense retrieval과 Sparse retrieval을 결합하여 검색 품질을 향상시키는 기법이다.

\begin{table}[H]
\centering
\caption{하이브리드 검색 기법}
\label{tab:hybrid-retrieval}
\small
\begin{tabular}{p{2cm}cp{5cm}p{5cm}}
\toprule
\textbf{연구} & \textbf{연도} & \textbf{주요 기여} & \textbf{한계점} \\
\midrule
RRF [17] & 2009 & 여러 검색 결과의 순위를 역수 합산(Reciprocal Rank Fusion) → 간단하게 여러 방식 통합 & 질의 특성 무시 → 수치/개념 질의 동일 처리 \\
BEIR Benchmark [3] & 2021 & 18개 데이터셋 평가 → Hybrid(키워드+의미 결합)가 Dense만보다 우수 입증 & 농업 데이터셋 미포함 → 도메인 성능 검증 안됨 \\
\bottomrule
\end{tabular}
\end{table}

\textbf{본 연구 대응:}
\begin{itemize}
\item \textbf{HybridDAT 3채널 융합} → Dense, Sparse, PathRAG 인과관계 그래프 검색을 통합
\item \textbf{DAT (Dynamic Alpha Tuning)} → 질의 특성에 따른 적응형 가중치 조정, 수치/개념 질의 최적화
\item 농업 도메인 수치 정보(온도, EC, pH 등) 매칭 최적화 → 온톨로지 매칭으로 부분 대응
\end{itemize}

\subsection{농업 도메인 지식 기반의 AI 접근 방법}

농업 도메인에서 AI를 효과적으로 적용하기 위해서는 도메인 지식의 체계적 표현과 활용이 필수적이다.

\subsubsection{농업 온톨로지 및 지식 그래프}

\begin{table}[H]
\centering
\caption{농업 온톨로지 및 지식 그래프}
\label{tab:agriculture-ontology}
\small
\begin{tabular}{p{2cm}cp{5cm}p{5cm}}
\toprule
\textbf{연구} & \textbf{연도} & \textbf{주요 기여} & \textbf{한계점} \\
\midrule
Bhuyan et al. [18] & 2021 & 농업 개념을 래티스(lattice, 계층 격자) 구조로 정리 → 작물-환경-병해 관계 체계화 & 지식 표현만 → 실제 검색할 때 활용 방법 없음 \\
Ahmadzai et al. [19] & 2024 & 텍스트에서 RE(Relation Extraction) 자동 구축 → 수작업 없이 지식 구조 생성 & RAG와 연결 안됨 → 검색 품질 향상에 기여 못함 \\
Cornei et al. [20] & 2024 & 센서 데이터를 시간순으로 표현 → 환경 변화 흐름 모델링 & 구조가 복잡함 → 실시간 검색에 부적합 \\
CropDP-KG (Yan) [21] & 2025 & NER+RE로 병해충 지식 13,840개 항목 자동 구축 → 대규모 KG 구축법 제시 & 학습 데이터 수만 건 필요 → 소규모 프로젝트 적용 어려움 \\
Tomato KG (Wang) [22] & 2024 & Stanford 온톨로지 방법론 6단계 적용 → 체계적 지식그래프 구축 검증 & 토마토만 다룸 → 여러 작물 재배 농가 활용 불가 \\
AgriKG (Chen) [23] & 2019 & 농업 지식그래프 활용 사례 → 추천/진단 등 응용 가능성 제시 & 검색 시스템과 연결 안됨 → 지식 있어도 찾기 어려움 \\
\bottomrule
\end{tabular}
\end{table}

\subsubsection{인과관계 추출 및 지식 연결}

\begin{table}[H]
\centering
\caption{인과관계 추출 및 지식 연결}
\label{tab:causal-extraction-appendix}
\small
\begin{tabular}{p{2.5cm}cp{5cm}p{4.5cm}}
\toprule
\textbf{연구} & \textbf{연도} & \textbf{주요 기여} & \textbf{한계점} \\
\midrule
Yang et al. [24] & 2022 & 인과관계 추출(Causal RE) 방법 분류 (규칙/ML/DL) → 방법 선택 가이드 제공 & RAG 통합 미고려 → 추출해도 검색에 활용 안됨 \\
Semi-sup. Agri RE [25] & 2023 & 규칙 기반 인과관계 추출 → GPU 없이 정밀도 86\% 달성 & 한 문장 안에서만 추출 → 서로 다른 문서 간 연결 불가 \\
CaEXR (Liu) [26] & 2024 & 단어 쌍 네트워크로 인과관계 공동 추출 → 복잡한 관계도 F1 82\% & GPU 필수 → 8GB RAM 엣지에서 실행 불가 \\
Sitokonstantinou et al. [27] & 2024 & CO2/온도가 수확량에 미치는 영향 수치화 → 환경 요인별 기여도 분석 & 수치 데이터만 분석 → 텍스트 기반 "원인-해결" 추론 못함 \\
\bottomrule
\end{tabular}
\end{table}

\textbf{본 연구 대응:}
\begin{itemize}
\item \textbf{HybridGraphBuilder 자동 그래프 구축} → 규칙 기반 + LLM 기반 하이브리드 인과관계 추출
\item \textbf{농업 온톨로지 기반 엔티티 추출 가이드} → 6개 유형(crop/disease/environment/practice/nutrient/stage) 도메인 특화
\item 검색 단계에서 온톨로지 직접 매칭 → 쿼리의 작물/환경/병해 개념을 즉시 인식
\item llama.cpp 통합으로 로컬 LLM 사용 → GPU 없이 CPU만으로 실행 가능
\end{itemize}

\subsection{검색 다양성 및 RAG 후처리}

RAG 시스템에서 검색된 문서들을 그대로 LLM에 전달하면 중복된 정보로 인해 답변 품질이 저하된다. 검색 다양성(diversity) 기법은 RAG 후처리 단계에서 관련성과 다양성의 균형을 맞춰 최종 컨텍스트를 선별한다.

\begin{table}[H]
\centering
\caption{검색 다양성 및 RAG 후처리}
\label{tab:diversity-reranking}
\small
\begin{tabular}{p{2cm}cp{5cm}p{5cm}}
\toprule
\textbf{연구} & \textbf{연도} & \textbf{주요 기여} & \textbf{한계점} \\
\midrule
MMR [28] & 1998 & MMR(Maximal Marginal Relevance) → 관련성 높은 문서 중 서로 다른 내용만 선택 & 다양성 조절 파라미터($\lambda$) 고정 → 작물별 중복 제거 같은 세부 요구 반영 불가 \\
VRSD [29] & 2024 & 새로운 다양성 선택 알고리즘 → MMR 대비 우수 & 일반 데이터로만 테스트 → 농업에서 효과 미검증 \\
SMMR [30] & 2025 & 샘플링 기반 MMR → 동일 품질에 로그 속도 향상 & 작물 구분 없음 → "와사비 질문에 상추 문서" 섞여 나옴 \\
\bottomrule
\end{tabular}
\end{table}

\textbf{본 연구 대응:}
\begin{itemize}
\item \textbf{PathRAG 인과관계 경로 탐색} → BFS 기반 2-hop 탐색으로 관련 정보 확장, 중복 감소
\item 작물별 검색 필터링: 질문 작물과 일치 시 보너스, 불일치 시 패널티 → "와사비 질문에 상추 문서" 문제 해결 (초기 탐색 후 성능 저하로 제거)
\item 메모리 적응형 리랭커 선택 → 가용 RAM에 따라 최적 품질 제공
\end{itemize}

\subsection{온디바이스 엣지 환경}

농업 현장에서 RAG 시스템을 실용화하려면 클라우드 의존 없이 저사양 엣지 디바이스에서 실행 가능해야 한다.

\subsubsection{LLM 경량화 및 배포}

\begin{table}[H]
\centering
\caption{LLM 경량화 및 배포}
\label{tab:llm-lightweight}
\small
\begin{tabular}{p{2.5cm}cp{5cm}p{4.5cm}}
\toprule
\textbf{연구} & \textbf{연도} & \textbf{주요 기여} & \textbf{한계점} \\
\midrule
Edge LLM Survey [31] & 2025 & 모델 압축 기법 정리 (양자화/지식증류/프루닝) → 엣지 배포 전략 가이드 & LLM만 다룸 → 검색+답변 전체 RAG 시스템 통합 방법 없음 \\
llama.cpp [32] & 2024 & GGUF 양자화(Q4\_K\_M 등) → 4B 모델 기준 8GB→2.5GB (약 70\%↓) & 생성만 지원 → 문서 검색 기능은 별도 구축 필요 \\
Model2Vec [33] & 2024 & 지식 증류 → 256차원 정적 벡터, 속도 100-400배↑ & 품질 5-10\%↓ → 정밀 검색 필요한 농업 질의에 부정확 \\
EmbeddingGemma [34] & 2025 & 308M 파라미터(약 600MB)로 1B급 성능 달성 → 적은 자원으로 고품질 벡터 검색 & 농업 데이터 벤치마크 없음 → 도메인 성능 보장 안됨 \\
\bottomrule
\end{tabular}
\end{table}

\subsubsection{엣지 RAG 시스템}

\begin{table}[H]
\centering
\caption{엣지 RAG 시스템}
\label{tab:edge-rag-systems}
\small
\begin{tabular}{p{2cm}cp{5cm}p{5cm}}
\toprule
\textbf{연구} & \textbf{연도} & \textbf{주요 기여} & \textbf{한계점} \\
\midrule
EdgeRAG [9] & 2024 & 계층적 인덱싱 → 필요한 임베딩만 로드하여 메모리 제약 해결 & Dense 검색만 지원 → 수치 검색 약함, 도메인 지식 활용 안함 \\
\bottomrule
\end{tabular}
\end{table}

\subsubsection{스마트팜 엣지 컴퓨팅}

\begin{table}[H]
\centering
\caption{스마트팜 엣지 컴퓨팅}
\label{tab:smartfarm-edge}
\small
\begin{tabular}{p{2.5cm}cp{5cm}p{4.5cm}}
\toprule
\textbf{연구} & \textbf{연도} & \textbf{주요 기여} & \textbf{한계점} \\
\midrule
FarmBeats (MS) [35] & 2017 & 농업 IoT 아키텍처 설계 → 센서 데이터 수집/전송 체계화 & 센서 데이터만 처리 → 텍스트 지식 검색/LLM 기능 없음 \\
Smart Farming Review [36] & 2020 & 농업 데이터 관리 전략 종합 리뷰 → Agriculture 5.0 로드맵 제시 & 데이터 관리 중심 → 실시간 추론/RAG 미다룸 \\
Farm-LightSeek [37] & 2025 & 경량 LLM 기반 멀티모달 IoT 분석 → 엣지에서 크로스모달 추론 & 이미지+센서 중심 → 텍스트 문서 기반 RAG 미포함 \\
\bottomrule
\end{tabular}
\end{table}

\textbf{본 연구 대응:}
\begin{itemize}
\item \textbf{llama.cpp Q4\_K\_M 양자화} → 8GB RAM 환경에서 전체 RAG 파이프라인 배포
\item \textbf{FAISS mmap 기반 인덱스 로딩} → 필요한 부분만 로드하여 대용량 인덱스도 저메모리에서 사용 가능
\item \textbf{메모리 적응형 리랭커 선택} → 가용 RAM 감지하여 리랭커 자동 선택, 자원에 맞춰 최선의 품질 제공
\item \textbf{HybridDAT 엣지 최적화} → 로컬 LLM으로 그래프 구축 및 3채널 하이브리드 검색 수행
\item 텍스트 질문-답변 RAG 지원 → 센서/이미지만 처리하던 기존 스마트팜 엣지에 자연어 질의 기능 추가
\end{itemize}

\subsection{연구 공백 및 기여 요약}

\begin{table}[H]
\centering
\caption{연구 공백 및 본 연구의 기여}
\label{tab:research-gap-contribution}
\small
\begin{tabular}{p{2.5cm}p{5.5cm}p{5.5cm}}
\toprule
\textbf{영역} & \textbf{기존 연구 한계} & \textbf{본 연구 대응} \\
\midrule
\textbf{Graph RAG} & LightRAG: 범용 엔티티 타입, 도메인 특화 없음 & HybridDAT: Dense+Sparse+PathRAG 3채널 융합, 농업 엔티티 타입 6종, 온톨로지 통합 \\
\textbf{Edge Deployment} & EdgeRAG: 범용 최적화, AgroMetLLM: 특정 태스크 한정 & llama.cpp Q4\_K\_M + FAISS mmap + 메모리 적응형 리랭커 → 8GB RAM 타겟 \\
\textbf{Agricultural KG} & CropDP-KG, AHR-RAG: 대규모 학습 데이터 필요 & HybridGraphBuilder 자동 구축 + 경량 온톨로지 \\
\textbf{Evaluation} & IR metrics: Ground Truth 의존, 고비용 어노테이션 & RAGAS reference-free + 로컬 LLM → 평가 비용 최소화 \\
\textbf{검색 다양성} & 작물 구분 없음 → 다른 작물 문서 섞임 & PathRAG 인과관계 경로 탐색 + 온톨로지 매칭 \\
\bottomrule
\end{tabular}
\end{table}

\subsection{RAG 평가 및 벤치마크}

RAG 시스템의 성능 평가는 검색 품질(Retrieval)과 생성 품질(Generation) 두 측면에서 이루어진다.

\subsubsection{전통적 평가 메트릭}

\begin{table}[H]
\centering
\caption{전통적 RAG 평가 메트릭}
\label{tab:traditional-metrics}
\small
\begin{tabular}{lll}
\toprule
\textbf{단계} & \textbf{메트릭} & \textbf{설명} \\
\midrule
\multirow{4}{*}{\textbf{검색 (IR)}} & Precision@K, Recall@K & 상위 K개 결과의 정밀도/재현율 \\
& MRR & 첫 번째 관련 문서의 역순위 평균 \\
& NDCG & 순위 품질 평가 (순위별 가중치) \\
& MAP & 평균 정밀도 \\
\midrule
\multirow{4}{*}{\textbf{생성 (QA)}} & Exact Match (EM) & 정답과 정확히 일치 여부 \\
& F1 Score & 토큰 수준 정밀도/재현율 조화평균 \\
& ROUGE & n-gram 기반 재현율 (요약 평가용) \\
& BLEU & n-gram 기반 정밀도 (번역 평가용) \\
\midrule
\textbf{의미 유사도} & BERTScore & 임베딩 기반 의미 유사도 \\
\bottomrule
\end{tabular}
\end{table}

\subsubsection{LLM-as-Judge 기반 Reference-free 평가}

\begin{table}[H]
\centering
\caption{LLM-as-Judge 기반 평가 방법}
\label{tab:llm-as-judge}
\small
\begin{tabular}{p{1.5cm}ccp{5cm}p{3cm}}
\toprule
\textbf{연구} & \textbf{연도} & \textbf{학회} & \textbf{주요 기여} & \textbf{한계점} \\
\midrule
\textbf{RAGAS} [38] & 2024 & EACL & Ground Truth 없이 Faithfulness, Answer Relevancy, Context Precision 측정 → RAG 평가 de facto 표준 & LLM judge 품질에 의존 \\
\textbf{ARES} [39] & 2024 & NAACL & Synthetic QA 자동 생성 + Fine-tuned LLM Judges & LLM API 비용 \\
\bottomrule
\end{tabular}
\end{table}

\subsubsection{최신 RAG 벤치마크 (2024--2025)}

\begin{table}[H]
\centering
\caption{최신 RAG 벤치마크 (2024--2025)}
\label{tab:rag-benchmarks}
\small
\begin{tabular}{p{2.5cm}cp{2cm}p{4cm}p{3cm}}
\toprule
\textbf{벤치마크} & \textbf{연도} & \textbf{규모} & \textbf{특징} & \textbf{한계점} \\
\midrule
\textbf{RAGBench} [40] & 2024 & 69K 예제 & 산업별 RAG 평가 지원 & 농업 미포함 \\
\textbf{CRAG} [41] & 2024 & 4,409 QA & KDD Cup 2024, 웹 검색 기반 현실적 시나리오 & 일반 도메인 \\
\textbf{GraphRAG-Bench} [42] & 2025 & -- & NeurIPS 2025, Graph 구조 활용 효과 정량화 & 최신 (적용 사례 제한적) \\
\textbf{AgXQA} [43] & -- & -- & 농업 기술 Q\&A 데이터셋 & 영어 중심, 한국어 미지원 \\
\bottomrule
\end{tabular}
\end{table}

\subsubsection{RAGAS 메트릭 상세}

\begin{table}[H]
\centering
\caption{RAGAS 메트릭 상세}
\label{tab:ragas-metrics}
\small
\begin{tabular}{p{3cm}p{2cm}p{2.5cm}p{5.5cm}}
\toprule
\textbf{메트릭} & \textbf{Ground Truth} & \textbf{평가 대상} & \textbf{설명} \\
\midrule
\textbf{Faithfulness} & 불필요 & Generation & 답변이 검색된 context에 충실한가 (환각 여부) \\
\textbf{Answer Relevancy} & 불필요 & Generation & 답변이 질문에 관련 있는가 \\
\textbf{Context Precision} & 불필요 & Retrieval & 검색된 문서 중 관련 문서 비율 \\
\textbf{Context Recall} & 선택적 & Retrieval & 답변 생성에 필요한 정보가 context에 있는가 \\
\textbf{Answer Correctness} & 필요 & Generation & 답변이 정답과 일치하는가 \\
\bottomrule
\end{tabular}
\end{table}

\textbf{본 연구 대응:}
\begin{itemize}
\item \textbf{RAGAS 기반 Reference-free 평가} → Ground Truth 의존성 해소, 도메인 특화 데이터셋 부재 문제 우회
\item \textbf{로컬 LLM 사용} → LLM API 비용 최소화
\item \textbf{IR 메트릭 + RAGAS 조합}: Recall@K, MRR, NDCG (검색) + Faithfulness, Answer Relevancy (생성) → 재현 가능한 평가 체계
\item 한국어 스마트팜 도메인 특화 벤치마크 부재 → 자체 평가 데이터셋 구축 및 RAGAS 활용
\end{itemize}

\subsection{참고문헌}

\begin{enumerate}
\item Lewis, P., et al. (2020). "Retrieval-Augmented Generation for Knowledge-Intensive NLP Tasks." \textit{NeurIPS 2020}.
\item Karpukhin, V., et al. (2020). "Dense Passage Retrieval for Open-Domain Question Answering." \textit{EMNLP 2020}.
\item Thakur, N., et al. (2021). "BEIR: A Heterogeneous Benchmark for Zero-shot Evaluation of Information Retrieval Models." \textit{NeurIPS Datasets and Benchmarks 2021}.
\item Gao, Y., et al. (2024). "Retrieval-Augmented Generation for Large Language Models: A Survey." arXiv:2312.10997.
\item Peng, B., et al. (2024). "Graph Retrieval-Augmented Generation: A Survey." arXiv:2408.08921.
\item Edge, D., et al. (2024). "From Local to Global: A Graph RAG Approach to Query-Focused Summarization." arXiv:2404.16130.
\item Guo, Z., et al. (2024). "LightRAG: Simple and Fast Retrieval-Augmented Generation." arXiv:2410.05779.
\item Yang, Y., et al. (2024). "Cluster-based Partial Dense Retrieval Fused with Sparse Text Retrieval." \textit{SIGIR 2024}.
\item Seemakhupt, K., et al. (2024). "EdgeRAG: Online-Indexed RAG for Edge Devices." arXiv:2412.21023.
\item Gong, R., \& Li, X. (2025). "The Application Progress and Research Trends of Knowledge Graphs and Large Language Models in Agriculture." \textit{Computers and Electronics in Agriculture}, 235, 110396.
\item Chen, B., et al. (2025). "PathRAG: Pruning Graph-based Retrieval Augmented Generation with Relational Paths." arXiv:2502.14902.
\item Wu, H., et al. (2025). "Crop GraphRAG: Pest and Disease Knowledge Base Q\&A System for Sustainable Crop Protection." \textit{Frontiers in Plant Science}.
\item Yang, J., et al. (2025). "Intelligent Q\&A Method for Crop Pests and Diseases Using LLM Augmented by Adaptive Hybrid Retrieval." \textit{Smart Agriculture}.
\item Li, X., et al. (2025). "ReG-RAG: A Large Language Model-based Question Answering Framework with Query Rewriting and Knowledge Graph Enhancement." \textit{Smart Agriculture}.
\item Ray, P. P., \& Pradhan, M. P. (2025). "AgroMetLLM: An Evapotranspiration and Agro-advisory System Using Localized Large Language Models in Resource-constrained Edge." \textit{Journal of Agrometeorology}, 27(3), 320-326.
\item Jiang, J., Yan, L., \& Liu, J. (2025). "Agricultural Large Language Model Based on Precise Knowledge Retrieval and Knowledge Collaborative Generation." \textit{Smart Agriculture}, 7(1), 20-32.
\item Cormack, G. V., Clarke, C. L. A., \& Büttcher, S. (2009). "Reciprocal Rank Fusion Outperforms Condorcet and Individual Rank Learning Methods." \textit{SIGIR 2009}.
\item Bhuyan, M., et al. (2021). "An Ontological Knowledge Representation for Smart Agriculture." \textit{IEEE BigData 2021}.
\item Ahmadzai, H., et al. (2024). "Innovative Agricultural Ontology Construction Using NLP." \textit{Engineering Science and Technology, an International Journal}, 53, 101699.
\item Cornei, L., Cornei, D., \& Foșalău, C. (2024). "An Ontology-Driven Solution for Capturing Spatial and Temporal Dynamics in Smart Agriculture." \textit{RCIS 2024, LNBIP vol 513}.
\item Yan, R., et al. (2025). "A Knowledge Graph for Crop Diseases and Pests in China (CropDP-KG)." \textit{Scientific Data}.
\item Wang, K., et al. (2024). "Research on the Construction of a Knowledge Graph for Tomato Leaf Pests and Diseases Based on NER Model." \textit{Frontiers in Plant Science}.
\item Chen, Y., et al. (2019). "AgriKG: An Agricultural Knowledge Graph and Its Applications." \textit{DASFAA 2019, LNCS vol 11448}.
\item Yang, J., et al. (2022). "A Survey on Extraction of Causal Relations from Natural Language Text." \textit{Knowledge and Information Systems}.
\item IEEE (2023). "Semi-Supervised Approach for Relation Extraction in Agriculture Documents."
\item Liu, C., et al. (2024). "CaEXR: A Joint Extraction Framework for Causal Relationships Based on Word-Pair Network." \textit{ICIC 2024, LNCS vol 14878}.
\item Sitokonstantinou, V., et al. (2024). "Causal Machine Learning for Sustainable Agroecosystems." arXiv:2408.13155.
\item Carbonell, J., \& Goldstein, J. (1998). "The Use of MMR, Diversity-Based Reranking for Reordering Documents and Producing Summaries." \textit{SIGIR 1998}.
\item Gao, H., \& Zhang, Y. (2024). "VRSD: Rethinking Similarity and Diversity for Retrieval in Large Language Models." arXiv:2407.04573.
\item Liakhnovich, K., et al. (2025). "SMMR: Sampling-Based MMR Reranking for Faster, More Diverse, and Balanced Recommendations and Retrieval." \textit{SIGIR 2025}.
\item Xu, Z., et al. (2025). "Sustainable LLM Inference for Edge AI: Evaluating Quantized LLMs." arXiv:2504.03360.
\item Gerganov, G. (2024). "llama.cpp: LLM Inference in C/C++." GitHub.
\item Tulkens, S., \& van Dongen, T. (2024). "Model2Vec: Fast State-of-the-Art Static Embeddings." GitHub.
\item Google Research (2025). "EmbeddingGemma: Best-in-Class Open Model for On-Device Embeddings."
\item Vasisht, D., et al. (2017). "FarmBeats: An IoT Platform for Data-Driven Agriculture." \textit{NSDI 2017}.
\item Saiz-Rubio, V., \& Rovira-Más, F. (2020). "From Smart Farming Towards Agriculture 5.0: A Review on Crop Data Management." \textit{Agronomy}, 10(2), 207.
\item Jiang, D., et al. (2025). "Farm-LightSeek: An Edge-centric Multimodal Agricultural IoT Data Analytics Framework with Lightweight LLMs." arXiv:2506.03168.
\item Es, S., et al. (2024). "RAGAS: Automated Evaluation of Retrieval Augmented Generation." \textit{EACL 2024}.
\item Saad-Falcon, J., et al. (2024). "ARES: An Automated Evaluation Framework for Retrieval-Augmented Generation Systems." \textit{NAACL 2024}.
\item Fröbe, M., et al. (2024). "RAGBench: Explainable Benchmark for Retrieval-Augmented Generation Systems." arXiv:2407.11005.
\item Yang, X., et al. (2024). "CRAG - Comprehensive RAG Benchmark." \textit{KDD Cup 2024}.
\item Wu, Y., et al. (2025). "GraphRAG-Bench: Benchmarking Graph-based Retrieval Augmented Generation." \textit{NeurIPS 2025}.
\item AgXQA: Agricultural Expert Question Answering Dataset.
\end{enumerate}
