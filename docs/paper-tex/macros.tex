% ==============================================================================
% Custom LaTeX Macros for ERA-SmartFarm-RAG Paper
% ==============================================================================
% This file defines custom commands and abbreviations for consistent
% terminology, formatting, and notation throughout the paper.
% ==============================================================================

% ------------------------------------------------------------------------------
% SECTION 1: PLACEHOLDERS
% ------------------------------------------------------------------------------

% TBD placeholder - renders as red "[TBD]" for unfinished sections
\newcommand{\tbd}{\textcolor{red}{\textbf{[TBD]}}}

% ------------------------------------------------------------------------------
% SECTION 2: DOMAIN TERMINOLOGY
% ------------------------------------------------------------------------------

% System and component names
\newcommand{\hybriddat}{HybridDAT}  % Hybrid Dynamic Alpha Tuning
\newcommand{\pathrag}{PathRAG}      % Path-based Graph RAG
\newcommand{\rrf}{RRF}              % Reciprocal Rank Fusion
\newcommand{\ragas}{RAGAS}          % RAG Assessment

% Retrieval channels
\newcommand{\densechannel}{Dense Channel}
\newcommand{\sparsechannel}{Sparse Channel}
\newcommand{\graphchannel}{Graph Channel}

% ------------------------------------------------------------------------------
% SECTION 3: TECHNICAL ABBREVIATIONS
% ------------------------------------------------------------------------------

% Core technologies
\newcommand{\rag}{RAG}   % Retrieval-Augmented Generation
\newcommand{\llm}{LLM}   % Large Language Model
\newcommand{\iot}{IoT}   % Internet of Things

% Evaluation frameworks
\newcommand{\mrr}{MRR}   % Mean Reciprocal Rank

% ------------------------------------------------------------------------------
% SECTION 4: KOREAN TERMINOLOGY (Optional)
% ------------------------------------------------------------------------------
% Uncomment if bilingual terms are needed in the paper

% \newcommand{\smartfarm}{스마트팜}     % Smart Farm
% \newcommand{\rda}{농촌진흥청}          % Rural Development Administration

% ------------------------------------------------------------------------------
% SECTION 5: MATHEMATICAL NOTATION
% ------------------------------------------------------------------------------

% Score notation for retrieval channels
% Usage: \score{d} for score_d, \score{s} for score_s, etc.
\newcommand{\score}[1]{\text{score}_{#1}}

% Alpha parameters for dynamic weighting
\newcommand{\alphadense}{\alpha_d}     % Alpha for dense channel
\newcommand{\alphasparse}{\alpha_s}    % Alpha for sparse channel
\newcommand{\alphapath}{\alpha_p}      % Alpha for PathRAG channel

% Beta parameters (if needed for future extensions)
\newcommand{\betadense}{\beta_d}
\newcommand{\betasparse}{\beta_s}

% Final fusion score
% Usage: \fusionscore produces: final = α_d×D + α_s×S + α_p×P
\newcommand{\fusionscore}{\text{final} = \alphadense \times D + \alphasparse \times S + \alphapath \times P}

% ------------------------------------------------------------------------------
% SECTION 6: REFERENCE SHORTCUTS
% ------------------------------------------------------------------------------

% Figure references
\newcommand{\figref}[1]{Figure~\ref{fig:#1}}
\newcommand{\figsref}[2]{Figures~\ref{fig:#1} and~\ref{fig:#2}}

% Table references
\newcommand{\tabref}[1]{Table~\ref{tab:#1}}
\newcommand{\tabsref}[2]{Tables~\ref{tab:#1} and~\ref{tab:#2}}

% Section references
\newcommand{\secref}[1]{Section~\ref{sec:#1}}
\newcommand{\secsref}[2]{Sections~\ref{sec:#1} and~\ref{sec:#2}}

% Equation references
% Note: \eqref is already defined by amsmath package
% \newcommand{\eqref}[1]{Equation~\ref{eq:#1}}
\newcommand{\myeqref}[1]{Equation~\ref{eq:#1}}

% ------------------------------------------------------------------------------
% SECTION 7: FORMATTING HELPERS
% ------------------------------------------------------------------------------

% Inline code snippets (defined in preamble.tex)
% \newcommand{\code}[1]{\texttt{#1}}

% File paths and system identifiers
\newcommand{\filepath}[1]{\texttt{#1}}

% Metrics and performance numbers
% Usage: \metric{Recall@4} produces: \textit{Recall@4}
\newcommand{\metric}[1]{\textit{#1}}

% Model names
\newcommand{\modelname}[1]{\textsc{#1}}

% ------------------------------------------------------------------------------
% SECTION 8: EXPERIMENTAL NOTATION
% ------------------------------------------------------------------------------

% Baseline configurations
\newcommand{\denseonly}{Dense-only}
\newcommand{\sparseonly}{Sparse-only}
\newcommand{\naivehybrid}{Naive Hybrid}
\newcommand{\woontology}{w/o Ontology}
\newcommand{\wodynamic}{w/o Dynamic Alpha}
\newcommand{\wopathrag}{w/o PathRAG}

% Evaluation metrics
\newcommand{\recallk}[1]{Recall@#1}
\newcommand{\precisionk}[1]{Precision@#1}

% ------------------------------------------------------------------------------
% SECTION 9: DATASET AND BENCHMARKS
% ------------------------------------------------------------------------------

% Dataset identifiers
\newcommand{\agriqa}{AgriQA}
\newcommand{\datasetsize}{220}
\newcommand{\corpussize}{400}

% Quality scores
\newcommand{\rougel}{ROUGE-L}
\newcommand{\groundedness}{Groundedness}

% ==============================================================================
% END OF MACROS
% ==============================================================================
