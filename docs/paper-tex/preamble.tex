% preamble.tex - LaTeX preamble for smartfarm RAG system paper
% Supports Korean typesetting, math, tables, graphics, and code listings

% ========================================
% Document Class and Layout
% ========================================
\usepackage[a4paper, margin=2.5cm]{geometry}
\usepackage{fancyhdr}
\setlength{\headheight}{14pt}
\pagestyle{fancy}
\fancyhf{}
\fancyhead[L]{\leftmark}
\fancyhead[R]{\thepage}
\renewcommand{\headrulewidth}{0.4pt}

% Line spacing
\usepackage{setspace}
\setstretch{1.2}

% ========================================
% Korean Typesetting (XeLaTeX)
% ========================================
\usepackage{kotex}
% Korean font setup (uncomment and adjust as needed)
% \setmainfont{Noto Serif CJK KR}
% \setsansfont{Noto Sans CJK KR}
% \setmonofont{Noto Sans Mono CJK KR}

% ========================================
% Math Packages
% ========================================
\usepackage{amsmath}
\usepackage{amssymb}
\usepackage{amsthm}
\usepackage{mathtools}
\usepackage{algorithm}
\usepackage{algpseudocode}

% Theorem environments
\theoremstyle{definition}
\newtheorem{definition}{Definition}[section]
\newtheorem{example}{Example}[section]
\newtheorem{remark}{Remark}[section]

\theoremstyle{plain}
\newtheorem{theorem}{Theorem}[section]
\newtheorem{lemma}[theorem]{Lemma}
\newtheorem{proposition}[theorem]{Proposition}

% ========================================
% Table Packages
% ========================================
\usepackage{booktabs}      % Professional quality tables
\usepackage{tabularx}      % Auto-width columns
\usepackage{multirow}      % Multi-row cells
\usepackage{makecell}      % Line breaks in cells
\usepackage{longtable}     % Multi-page tables
\usepackage{array}         % Extended column definitions

% Table column types
\newcolumntype{L}[1]{>{\raggedright\arraybackslash}p{#1}}
\newcolumntype{C}[1]{>{\centering\arraybackslash}p{#1}}
\newcolumntype{R}[1]{>{\raggedleft\arraybackslash}p{#1}}

% ========================================
% Graphics and Figures
% ========================================
\usepackage{graphicx}
\usepackage{subcaption}    % Subfigures
\usepackage{float}         % Better float control
\usepackage{wrapfig}       % Wrapped figures

% Graphics path
\graphicspath{{../figures/}{./figures/}}

% TikZ for diagrams
\usepackage{tikz}
\usetikzlibrary{positioning, shapes, arrows, arrows.meta, calc, fit, backgrounds}

% PGFPlots for scientific plots
\usepackage{pgfplots}
\pgfplotsset{compat=1.18}

% Color definitions for diagrams
\definecolor{diagramblue}{RGB}{52, 152, 219}
\definecolor{diagramgreen}{RGB}{46, 204, 113}
\definecolor{diagrampurple}{RGB}{155, 89, 182}
\definecolor{diagramorange}{RGB}{230, 126, 34}
\definecolor{diagramred}{RGB}{231, 76, 60}
\definecolor{diagramgray}{RGB}{149, 165, 166}

% ========================================
% Code Listings
% ========================================
\usepackage{listings}
\usepackage{xcolor}

% Define colors for code listings
\definecolor{codegreen}{rgb}{0,0.6,0}
\definecolor{codegray}{rgb}{0.5,0.5,0.5}
\definecolor{codepurple}{rgb}{0.58,0,0.82}
\definecolor{backcolour}{rgb}{0.95,0.95,0.92}

% Python listing style
\lstdefinestyle{pythonstyle}{
    backgroundcolor=\color{backcolour},
    commentstyle=\color{codegreen},
    keywordstyle=\color{magenta},
    numberstyle=\tiny\color{codegray},
    stringstyle=\color{codepurple},
    basicstyle=\ttfamily\footnotesize,
    breakatwhitespace=false,
    breaklines=true,
    captionpos=b,
    keepspaces=true,
    numbers=left,
    numbersep=5pt,
    showspaces=false,
    showstringspaces=false,
    showtabs=false,
    tabsize=2,
    frame=single,
    rulecolor=\color{black}
}

% JSON listing style
\lstdefinestyle{jsonstyle}{
    backgroundcolor=\color{backcolour},
    basicstyle=\ttfamily\footnotesize,
    breakatwhitespace=false,
    breaklines=true,
    captionpos=b,
    keepspaces=true,
    numbers=left,
    numbersep=5pt,
    showspaces=false,
    showstringspaces=false,
    showtabs=false,
    tabsize=2,
    frame=single,
    rulecolor=\color{black},
    stringstyle=\color{codepurple},
    numberstyle=\tiny\color{codegray}
}

\lstset{style=pythonstyle}

% ========================================
% References and Links
% ========================================
\usepackage{url}
\usepackage[bookmarks=true, unicode=true, colorlinks=true,
            linkcolor=blue, citecolor=blue, urlcolor=blue,
            pdfauthor={}, pdftitle={}, pdfsubject={}, pdfkeywords={}]{hyperref}

% Cleveref for smart references (load after hyperref)
\usepackage[capitalise, noabbrev]{cleveref}

% Custom cleveref names for Korean support
\crefname{figure}{그림}{그림}
\crefname{table}{표}{표}
\crefname{section}{절}{절}
\crefname{equation}{식}{식}
\crefname{theorem}{정리}{정리}
\crefname{lemma}{보조정리}{보조정리}
\crefname{definition}{정의}{정의}

% ========================================
% Bibliography Setup
% ========================================
\usepackage[backend=biber, style=numeric, sorting=nyt, maxbibnames=99]{biblatex}
% Note: bibliography resource is set in main.tex
% \addbibresource{bibliography/references.bib}

% ========================================
% Additional Utility Packages
% ========================================
\usepackage{enumitem}      % Customizable lists
\usepackage{caption}       % Custom captions
\usepackage{textcomp}      % Additional text symbols
\usepackage{gensymb}       % Generic symbols (degree, etc.)

% Caption setup
\captionsetup{
    labelfont=bf,
    textfont=it,
    font=small
}

% ========================================
% Custom Commands
% ========================================

% Math operators
\DeclareMathOperator*{\argmax}{argmax}
\DeclareMathOperator*{\argmin}{argmin}

% Convenience commands
\newcommand{\todo}[1]{\textcolor{red}{\textbf{TODO: #1}}}
\newcommand{\note}[1]{\textcolor{blue}{\textbf{Note: #1}}}

% Vector notation
\renewcommand{\vec}[1]{\mathbf{#1}}

% Code inline
\newcommand{\code}[1]{\texttt{#1}}

% ========================================
% End of Preamble
% ========================================
