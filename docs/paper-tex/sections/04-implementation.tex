\section{구현 (Implementation)}
\label{sec:impl}

\subsection{기술 스택}
\label{sec:impl-tech-stack}

시스템의 기술 스택은 서버 환경과 엣지 환경에 따라 달리 구성되었다.
표~\ref{tab:tech-stack}은 각 구성요소별 환경별 선택을 요약한다.

\begin{table}[h]
\centering
\caption{서버 환경과 엣지 환경별 기술 스택}
\label{tab:tech-stack}
\begin{tabular}{llll}
\toprule
\textbf{구성요소} & \textbf{서버 환경} & \textbf{엣지 환경} & \textbf{참조} \\
\midrule
Dense Retrieval & FAISS + Qwen3-Embedding-0.6B & FAISS + MiniLM-L6 & \cite{tulkens2024model2vec,google2025embeddinggemma} \\
 & (6억 파라미터, \textasciitilde1.2GB) & (2,200만 파라미터, \textasciitilde90MB) & \\
Sparse Retrieval & TF-IDF (scikit-learn) & TF-IDF (동일) & --- \\
지식 그래프 & 커스텀 그래프 (JSON) & 서브셋 그래프 & --- \\
LLM & llama.cpp (FP16/INT8) & llama.cpp (Q4\_K\_M) & \cite{gerganov2024llamacpp} \\
API & FastAPI + Docker & FastAPI (경량) & --- \\
오프라인 폴백 & --- & 캐시 + 규칙 기반 & \cite{seemakhupt2024edgerag,jiang2025farmlightseek} \\
\bottomrule
\end{tabular}
\end{table}

\subsection{핵심 모듈 구현}
\label{sec:impl-core-modules}

\subsubsection{HybridDATRetriever}
\label{sec:impl-hybrid-dat}

3채널 검색 융합과 후처리를 담당하는 핵심 리트리버이다.

\paragraph{주요 파라미터}
표~\ref{tab:hybrid-dat-params}는 HybridDATRetriever의 주요 파라미터와 설정값을 나타낸다.

\begin{table}[h]
\centering
\caption{HybridDATRetriever 주요 파라미터}
\label{tab:hybrid-dat-params}
\begin{tabular}{lll}
\toprule
\textbf{파라미터} & \textbf{값} & \textbf{설명} \\
\midrule
\texttt{use\_rrf} & true & RRF 기반 융합 활성화 \\
\texttt{rrf\_k} & 60 & RRF 파라미터 \\
\texttt{use\_dat} & true & 동적 가중치 조정 활성화 \\
\texttt{use\_ontology} & true & 온톨로지 매칭 활성화 \\
\texttt{use\_pathrag} & false & PathRAG 활성화 (선택적) \\
\bottomrule
\end{tabular}
\end{table}

\paragraph{동적 가중치 계산}
\texttt{dynamic\_alphas} 함수는 온톨로지 매칭 결과와 수치/단위 패턴을 분석하여 3채널 가중치를 반환한다.
LLM 호출 없이 규칙 기반으로 동작하여 엣지 환경에 최적화되어 있다.

\subsubsection{PathRAGRetriever (PathRAG-lite)}
\label{sec:impl-pathrag-lite}

PathRAG\cite{chen2025pathrag}의 경로 탐색 개념을 차용한 경량 구현이다.
원본 PathRAG의 relational path pruning 대신 BFS(Breadth-First Search, 너비 우선 탐색: 가까운 노드부터 순서대로 방문) 기반 단순화된 탐색을 수행한다.

\paragraph{탐색 전략}
\begin{itemize}
    \item \textbf{시작점}: 쿼리에서 매칭된 온톨로지 개념 노드 (예: ``와사비 고수온'' $\rightarrow$ \texttt{crop:와사비}, \texttt{env:고수온})
    \item \textbf{최대 깊이}: 2-hop (2번까지 엣지를 따라 이동, 기본값)
    \item \textbf{우선 탐색}: 인과관계 엣지(\texttt{causes}, \texttt{solved\_by})를 우선 탐색하여 원인$\rightarrow$결과$\rightarrow$해결책 문서 수집
\end{itemize}

\subsubsection{GraphBuilder}
\label{sec:impl-graph-builder}

문서 인제스트 시 인과관계 그래프를 자동 구축한다.

\paragraph{인과관계 패턴}
다음 패턴을 사용하여 문서에서 인과관계를 추출한다.

\begin{lstlisting}[language=Python, caption={Causal Pattern Definitions}, label={lst:cause-patterns}, escapechar=|]
# Causal pattern examples
CAUSE_PATTERNS = ["cause", "reason", "because", "if-then", "when high", "when low"]
EFFECT_PATTERNS = ["result", "effect", "symptom", "problem", "failure", "decline"]
SOLUTION_PATTERNS = ["solve", "respond", "method", "action", "manage", "prevent"]
\end{lstlisting}

\paragraph{엣지 생성 로직}
그래프 구축은 다음 단계로 수행된다.

\begin{enumerate}
    \item 각 문서의 인과관계 역할 탐지 (cause/effect/solution)
    \item 공통 키워드(작물, 환경요소, 병해, 상태) 추출
    \item 키워드 교집합이 존재하는 문서 쌍에 엣지 생성
\end{enumerate}

\subsubsection{EmbeddingRetriever}
\label{sec:impl-embedding-retriever}

FAISS(Facebook AI Similarity Search, 벡터 유사도 검색 라이브러리) 기반 Dense 검색을 담당한다.

\paragraph{특징}
\begin{itemize}
    \item \textbf{Lazy loading(지연 로딩)}: 시작 시가 아닌 첫 쿼리 시점에 모델 로드하여 초기 메모리 절약
    \item \textbf{L2 정규화}: 임베딩을 L2 정규화하여 코사인 유사도 검색 수행 (벡터 길이 1로 맞춰 방향만 비교)
    \item \textbf{mmap 지원}: memory-mapped file(파일을 메모리에 통째로 올리지 않고 필요한 부분만 로드)을 지원하여 대용량 인덱스도 저메모리에서 사용 가능
\end{itemize}

\subsection{메모리 적응형 리랭킹}
\label{sec:impl-adaptive-reranking}

리랭킹(reranking)은 검색된 문서들을 다시 정렬하여 가장 관련 높은 문서를 상위로 올리는 과정이다.
런타임 가용 메모리에 따라 리랭커를 동적으로 선택한다.
표~\ref{tab:adaptive-reranking}는 메모리 가용량에 따른 리랭커 선택 전략을 나타낸다.

\begin{table}[h]
\centering
\caption{메모리 적응형 리랭커 선택}
\label{tab:adaptive-reranking}
\begin{tabular}{lll}
\toprule
\textbf{가용 RAM} & \textbf{리랭커} & \textbf{설명} \\
\midrule
< 0.8GB & none & 리랭킹 비활성화 (검색 결과 그대로 사용) \\
0.8GB \textasciitilde{} 1.5GB & LLM-lite & llama.cpp 기반 경량 리랭킹 (쿼리-문서 관련성 재평가) \\
$\geq$ 1.5GB & BGE & BGE-reranker-v2-m3 (BERT 기반 고품질 리랭킹) \\
\bottomrule
\end{tabular}
\end{table}

\paragraph{설정 파라미터}
\begin{itemize}
    \item \texttt{AUTO\_RERANK\_MIN\_RAM\_GB}: 0.8
    \item \texttt{AUTO\_BGE\_MIN\_RAM\_GB}: 1.5
    \item \texttt{AUTO\_BGE\_MIN\_VRAM\_GB}: 1.5 (GPU 사용 시)
\end{itemize}

\subsection{인덱스 영속화}
\label{sec:impl-index-persistence}

오프라인 환경 지원을 위해 인덱스(검색용 데이터 구조)를 파일로 저장/로드한다.
시스템 재시작 시 문서를 다시 처리하지 않고 저장된 인덱스를 바로 로드한다.
표~\ref{tab:index-files}는 저장되는 인덱스 파일의 종류와 형식을 나타낸다.

\begin{table}[h]
\centering
\caption{인덱스 영속화 파일}
\label{tab:index-files}
\begin{tabular}{lll}
\toprule
\textbf{파일} & \textbf{내용} & \textbf{형식} \\
\midrule
\texttt{dense.faiss} & 문서 임베딩 벡터 인덱스 & Binary (mmap 가능, 부분 로드) \\
\texttt{dense\_docs.jsonl} & 문서 텍스트 및 메타데이터 & JSON Lines (한 줄에 한 문서) \\
\texttt{sparse.pkl} & TF-IDF 키워드 빈도 행렬 & Pickle (Python 직렬화) \\
\bottomrule
\end{tabular}
\end{table}

\subsection{그래프 스키마}
\label{sec:impl-graph-schema}

CropDP-KG\cite{yan2025cropdpkg}와 AgriKG\cite{chen2019agrikg}의 스키마 설계를 참조하여 구성하였다.

\paragraph{노드 타입}
\texttt{practice}(문서), \texttt{crop}, \texttt{env}, \texttt{disease}, \texttt{nutrient}, \texttt{stage}

\paragraph{엣지 타입}
표~\ref{tab:edge-types}는 정의된 엣지 타입과 의미를 나타낸다.

\begin{table}[h]
\centering
\caption{그래프 엣지 타입}
\label{tab:edge-types-impl}
\begin{tabular}{llp{5cm}l}
\toprule
\textbf{타입} & \textbf{의미} & \textbf{설명} & \textbf{참조} \\
\midrule
\texttt{recommended\_for} & 작물 $\rightarrow$ 실천 & 특정 작물에 권장되는 실천 & \cite{chen2019agrikg} \\
\texttt{associated\_with} & 병해 $\rightarrow$ 실천 & 병해와 관련된 실천 & \cite{yan2025cropdpkg} \\
\texttt{mentions} & 실천 $\rightarrow$ 개념 & 문서에서 언급된 개념 & \cite{ahmadzai2024innovative} \\
\texttt{causes} & 실천 $\rightarrow$ 실천 & 인과관계: 원인 문서 & \cite{yang2022causal,ieee2023semisupervised} \\
\texttt{solved\_by} & 실천 $\rightarrow$ 실천 & 인과관계: 해결책 문서 & \cite{yang2022causal,ieee2023semisupervised} \\
\bottomrule
\end{tabular}
\end{table}

\subsection{엣지 배포 사양}
\label{sec:impl-edge-deployment}

표~\ref{tab:edge-deployment}는 환경별 최소 및 권장 사양과 지원 기능을 나타낸다.

\begin{table}[h]
\centering
\caption{환경별 배포 사양}
\label{tab:edge-deployment}
\begin{tabular}{llll}
\toprule
\textbf{환경} & \textbf{최소 사양} & \textbf{권장 사양} & \textbf{지원 기능} \\
\midrule
서버 & 32GB RAM, GPU & 64GB RAM, RTX 4090 & 전체 기능 \\
엣지 게이트웨이 & 8GB RAM, CPU & 16GB RAM, CPU/NPU & RAG + Q4 LLM \\
저사양 엣지 & 4GB RAM & 8GB RAM & 검색 전용 \\
IoT 노드 & 512MB RAM & 1GB RAM & 센서 + 규칙 \\
\bottomrule
\end{tabular}
\end{table}

\subsection{EdgeRAG와의 구현 비교}
\label{sec:impl-edgerag-comparison}

표~\ref{tab:edgerag-comparison}은 EdgeRAG\cite{seemakhupt2024edgerag}와 본 시스템의 구현 특성을 비교한다.

\begin{table}[h]
\centering
\caption{EdgeRAG와 본 시스템의 구현 비교}
\label{tab:edgerag-comparison-impl}
\begin{tabular}{lp{5.5cm}p{5.5cm}}
\toprule
\textbf{구분} & \textbf{EdgeRAG\cite{seemakhupt2024edgerag}} & \textbf{본 시스템} \\
\midrule
최적화 초점 & 범용 메모리 최적화 & 도메인 특화 + 엣지 배포 \\
인덱싱 전략 & 온라인 계층적 인덱싱 & 오프라인 사전 인덱싱 + FAISS mmap \\
검색 채널 & 단일 Dense & Dense + Sparse + PathRAG 3채널 \\
그래프 활용 & 없음 & 인과관계 그래프 (\texttt{causes}, \texttt{solved\_by}) \\
도메인 지식 & 범용 & 농업 온톨로지 6개 유형 \\
메모리 절감 & 계층적 로딩으로 50\%$\downarrow$ & 양자화(Q4\_K\_M)로 75\%$\downarrow$ + mmap \\
오프라인 지원 & 제한적 & Sparse 검색 + 캐시 폴백 \\
품질 향상 & 메모리 효율 우선 & 작물 필터링, 중복 제거 \\
\bottomrule
\end{tabular}
\end{table}

\paragraph{핵심 차별점}
\begin{enumerate}
    \item \textbf{도메인 특화}: EdgeRAG가 범용 메모리 최적화에 집중하는 반면, 본 시스템은 농업 온톨로지와 인과관계 그래프를 활용하여 검색 품질 향상
    \item \textbf{멀티 채널}: 수치/단위 정보(EC, pH)의 정확한 매칭을 위한 Sparse 채널과 인과관계 추론을 위한 PathRAG 채널 유지
    \item \textbf{메모리 적응형}: 런타임 가용 메모리에 따른 동적 리랭커 선택
\end{enumerate}
