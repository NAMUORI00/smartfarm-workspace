\section{관련 연구 (Related Work)}
\label{sec:related}

본 연구는 RAG 시스템, 하이브리드 검색, 그래프 기반 지식 표현, 엣지 배포의 교차점에 위치한다. 본 장에서는 각 영역의 최근 발전과 한계를 검토하고, 연구 공백을 식별한다.

\subsection{RAG and Graph-based Retrieval}
\label{sec:related:rag-graph}

Retrieval-Augmented Generation(RAG)은 외부 지식 검색을 LLM 생성과 결합하여 환각을 줄이는 패러다임이다~\cite{lewis2020rag}. Gao et al.의 서베이~\cite{gao2024survey}는 RAG 발전을 Naive RAG(단순 검색-생성), Advanced RAG(쿼리 변환, 리랭킹), Modular RAG(컴포넌트 조합)로 분류하였다.

최근 그래프 기반 RAG가 주목받고 있다. \textbf{GraphRAG}~\cite{edge2024graphrag}는 커뮤니티 탐지로 문서를 클러스터링하여 전역 질의에 대응하나, LLM 기반 엔티티 추출에 높은 비용이 소요된다(1K 문서당 GPT-4 수천 회 호출, \$100+). \textbf{LightRAG}~\cite{guo2024lightrag}는 Dual-Level 검색(엔티티 수준 + 커뮤니티 수준)과 ego-network 기반 효율적 그래프 탐색으로 GraphRAG 대비 검색 효율성을 크게 개선하였으며, 로컬 LLM과의 통합이 용이하다. 그러나 범용 엔티티 타입(person, organization, location 등)을 사용하여 농업 등 특정 도메인에 대한 최적화가 필요하다. \textbf{PathRAG}~\cite{chen2025pathrag}는 관계 경로 기반 검색으로 중복을 줄이며 농업 포함 6개 도메인에서 우수한 성능을 보였다.

2025년 농업 도메인 RAG 연구가 본격화되었다. \textbf{Crop GraphRAG}~\cite{wu2025cropgraphrag}는 병해충 지식 그래프와 RAG를 결합하였고, \textbf{AHR-RAG}~\cite{yang2025intelligentqa}는 91만 트리플릿 KB로 복잡 질의에 대응하였다. \textbf{AgroMetLLM}~\cite{ray2025agrometllm}은 Raspberry Pi에서 양자화 LLM 기반 오프라인 농업 자문을 구현하였으나 증발산 예측에 특화되어 있다. 이들 연구는 엣지 환경에서의 범용 농업 Q\&A를 다루지 않는다.

\subsection{Hybrid Retrieval}
\label{sec:related:hybrid}

Dense retrieval은 의미적 유사성 매칭에 강하나 수치/단위 정보 매칭에 취약하고~\cite{karpukhin2020dense}, Sparse retrieval(BM25)은 그 반대 특성을 보인다. Hybrid retrieval은 두 방식을 융합하여 상호 보완한다~\cite{yang2024cluster}.

융합 전략으로 \textbf{Reciprocal Rank Fusion(RRF)}~\cite{cormack2009rrf}과 \textbf{Convex Combination}~\cite{yang2024cluster}이 널리 사용된다. Bruch et al.~\cite{yang2024cluster}은 RRF가 파라미터에 민감하며, Convex Combination이 in-domain/out-of-domain 모두에서 우수함을 보였다. SIGIR 2024의 클러스터 기반 연구는 Sparse 결과를 가이드로 활용한 융합 최적화를 제안하였다.

기존 하이브리드 검색은 Dense-Sparse 2채널 융합에 머물며, 그래프 기반 검색과의 통합은 제한적이다. LightRAG는 그래프 구조 내에서 벡터 검색을 수행하여 이러한 통합을 시도하나, 농업 도메인의 수치 정보(온도, EC, pH 등) 매칭에 대한 최적화는 부족하다.

\subsection{Agricultural Knowledge Systems}
\label{sec:related:agriculture}

농업 온톨로지 연구는 지식 표현에 초점을 맞춰왔다. Bhuyan et al.~\cite{bhuyan2021ontology}은 시공간 농업 데이터 추론을 위한 래티스 구조를 제안하였고, 스마트 농업 온톨로지~\cite{cornei2024ontology}와 NLP 기반 개발 방법론~\cite{ahmadzai2024innovative}이 발표되었다. \textbf{CropDP-KG}~\cite{yan2025cropdpkg}는 NER/RE로 13,840 엔티티와 21,961 관계를 구축하였으나, 수만 건의 학습 데이터와 레이블링 비용이 필요하다.

인과관계 추출 연구~\cite{yang2022causal,ieee2023semisupervised,liu2024caexr}는 문장 수준 관계 식별에 집중하며, 문서 간 인과관계 연결(``문서 A의 원인 → 문서 B의 해결책'')을 다루지 않는다. 딥러닝 기반 방식~\cite{liu2024caexr}은 GPU가 필수라 엣지 환경에서 실행이 불가능하다.

기존 농업 온톨로지는 검색 단계에서 직접 활용되지 않아 지식 정리가 검색 품질 향상에 기여하지 못한다.

\subsection{Edge Deployment for RAG}
\label{sec:related:edge}

엣지 LLM 배포를 위한 압축 기법으로 양자화, 지식 증류, 프루닝이 연구되고 있다~\cite{xu2025sustainable}. \textbf{llama.cpp}~\cite{gerganov2024llamacpp}는 GGUF 양자화로 CPU/저사양 GPU에서 LLM 추론을 가능하게 하며, Q4\_K\_M 양자화는 메모리를 약 70\% 절감한다.

\textbf{EdgeRAG}~\cite{seemakhupt2024edgerag}는 계층적 인덱싱과 선택적 로딩으로 메모리 50\%+ 감소를 달성하였으나, 도메인 특화 지식을 활용하지 않으며 단일 Dense 검색만 지원한다. \textbf{Model2Vec}~\cite{tulkens2024model2vec}는 100-400배 빠른 추론을 달성하나 품질이 5-10\% 하락한다.

스마트팜 엣지 연구로 \textbf{FarmBeats}~\cite{vasisht2017farmbeats}의 IoT 아키텍처, \textbf{Farm-LightSeek}~\cite{jiang2025farmlightseek}의 경량 CNN 병해충 분류가 있으나, 텍스트 Q\&A를 위한 RAG 기반 지식 검색의 엣지 배포는 다루지 않는다.

\subsection{RAG Evaluation}
\label{sec:related:evaluation}

전통적 RAG 평가는 검색 단계에서 Precision@K, Recall@K, MRR, NDCG, MAP 등 IR 메트릭을, 생성 단계에서 Exact Match(EM), F1 Score, ROUGE, BLEU 등을 사용한다. BERTScore는 임베딩 기반 의미 유사도를 측정하여 n-gram 한계를 보완한다. 그러나 이들 메트릭은 Ground Truth 구축에 높은 비용이 소요되며, 도메인 특화 데이터셋 부재 시 적용이 어렵다.

\textbf{RAGAS}~\cite{es2024ragas}는 LLM-as-Judge 기반 reference-free 평가 프레임워크로 Faithfulness, Answer Relevancy, Context Precision을 측정하며, EACL 2024에서 발표되어 RAG 평가의 de facto 표준으로 자리잡았다. \textbf{ARES}~\cite{saadfalcon2024ares}는 Synthetic QA 생성과 Fine-tuned LLM Judge를 결합하며 NAACL 2024에서 발표되었다.

2024-2025년 RAG 벤치마크가 다양화되었다. \textbf{RAGBench}~\cite{niu2024ragchecker}는 69K 예제로 산업별 RAG 평가를 지원하고, \textbf{CRAG}는 웹 검색 기반 4,409개 QA 쌍으로 현실적 시나리오를 평가한다. Graph RAG 특화 평가로 \textbf{GraphRAG-Bench}가 NeurIPS 2025에서 발표되어 그래프 구조 활용 효과를 정량화한다.

농업 도메인 벤치마크는 제한적이다. \textbf{AgXQA}는 농업 기술 Q\&A 데이터셋이나 영어 중심이며, 한국어 스마트팜 도메인 특화 벤치마크는 부재하다. 이에 본 연구는 RAGAS 기반 reference-free 평가와 표준 IR 메트릭을 조합하여 재현 가능한 평가 체계를 구축한다.

\subsection{Research Gap and Our Contributions}
\label{sec:related:gap}

Table~\ref{tab:comparison}은 기존 연구와 본 연구의 차별점을 요약한다.

\begin{table}[ht]
\centering
\caption{Comparison with Existing Approaches}
\label{tab:comparison}
\begin{tabular}{|p{2.5cm}|p{4cm}|p{3cm}|p{4cm}|}
\hline
\textbf{Aspect} & \textbf{Prior Work} & \textbf{Gap} & \textbf{Our Approach} \\
\hline
\textbf{Graph RAG} & LightRAG~\cite{guo2024lightrag}, PathRAG~\cite{chen2025pathrag} & 범용 엔티티 타입, 도메인 특화 없음 & HybridDAT: Dense+Sparse+PathRAG 3채널 융합, 농업 온톨로지 통합 \\
\hline
\textbf{Edge Deployment} & EdgeRAG~\cite{seemakhupt2024edgerag}, AgroMetLLM~\cite{ray2025agrometllm} & 범용 최적화, 특정 태스크 한정 & llama.cpp Q4\_K\_M + FAISS mmap, 8GB RAM 타겟 \\
\hline
\textbf{Agricultural KG} & CropDP-KG~\cite{yan2025cropdpkg}, AHR-RAG~\cite{yang2025intelligentqa} & 대규모 학습 데이터 필요 & HybridGraphBuilder (Rule + LLM 기반 인과관계 추출) \\
\hline
\textbf{Evaluation} & IR metrics with Ground Truth & 고비용 어노테이션 & RAGAS reference-free + 로컬 LLM \\
\hline
\end{tabular}
\end{table}

본 연구는 기존 Graph RAG 연구(LightRAG~\cite{guo2024lightrag}, PathRAG~\cite{chen2025pathrag})의 개념을 참고하되, \textbf{농업 도메인 특화 하이브리드 검색 시스템(HybridDAT)}을 독자적으로 설계한다:

\begin{enumerate}
\item \textbf{Dense-Sparse-Graph 3채널 융합 (HybridDATRetriever)}: RRF 기반 점수 융합과 DAT(Dynamic Alpha Tuning)를 통한 질의 적응형 가중치 조정. 농업 온톨로지 매칭으로 도메인 관련성 강화
\item \textbf{하이브리드 그래프 구축 (HybridGraphBuilder)}: 규칙 기반 패턴 매칭과 LLM 기반 인과관계 추출(CausalExtractor)을 결합하여 농업 지식 그래프 자동 구축. 농업 엔티티 6종(crop, disease, environment, practice, nutrient, stage) 지원
\item \textbf{엣지 환경 최적화}: llama.cpp Q4\_K\_M 양자화로 8GB RAM 환경 지원, FAISS mmap 기반 인덱스 로딩, 메모리 적응형 리랭커 선택
\item \textbf{Reference-free 평가}: RAGAS 기반 평가 파이프라인으로 Ground Truth 의존성 해소, 로컬 LLM으로 평가 비용 최소화
\end{enumerate}

Baseline 비교로 LightRAG~\cite{guo2024lightrag}와 직접 성능 비교를 수행하여 제안 시스템의 도메인 특화 효과를 검증한다. 초기에 규칙 기반 동적 가중치(Crop Filter, Semantic Dedup)를 탐색하였으나, 휴리스틱 설계의 일반화 어려움과 성능 저하로 제거하고 현재의 4-컴포넌트 구조(RRF, DAT, Ontology, PathRAG)로 정착하였다.
