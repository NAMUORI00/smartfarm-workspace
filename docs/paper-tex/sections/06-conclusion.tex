\section{결론 (Conclusion)}
\label{sec:conclusion}

% ---

\subsection{연구 요약}

본 연구는 스마트팜 도메인에 특화된 온디바이스 하이브리드 RAG 시스템을 제안하였다.

\textbf{주요 기여:}

\begin{enumerate}
    \item \textbf{3채널 검색 융합}: Dense-Sparse-PathRAG 검색 채널을 동적 가중치로 결합하여 질의 특성에 따른 최적 검색 수행
    \item \textbf{도메인 온톨로지}: 작물-환경-병해-영양소-생육단계-재배실천 6개 개념 유형으로 농업 도메인 특화 검색 품질 향상
    \item \textbf{인과관계 그래프}: 규칙 기반 패턴 매칭으로 원인→결과→해결책 인과 체인 구축 (LLM 비용 \$0)
    \item \textbf{엣지 배포 최적화}: Q4\_K\_M 양자화와 FAISS mmap으로 8GB RAM 엣지 디바이스에서 실시간 추론 지원
\end{enumerate}

\textbf{파일럿 스터디 성과:}

본 연구는 220개 QA 쌍을 활용한 파일럿 스터디로서, 소규모 데이터셋의 통계적 한계(MDE \textasciitilde{}4-5\%)를 인지하면서도 시스템 아키텍처의 타당성과 엣지 배포 가능성을 검증하였다. 특히 MiniLM 임베딩 모델 적용 시 25-40x 속도 향상을 달성하여 8GB RAM 환경에서의 실용적 운영 가능성을 확인하였다.

% ---

\subsection{향후 연구}

본 파일럿 스터디의 한계를 해결하기 위한 후속 연구 방향을 제시한다.

\subsubsection{단기 과제 (Short-term)}

\begin{enumerate}
    \item \textbf{데이터셋 확장}: QA 500개 이상 확보, 다중 작물(토마토, 딸기, 파프리카) 추가하여 통계적 검정력 강화
    \item \textbf{전문가 검증}: 50-100개 샘플에 대한 농업 전문가 품질 평가로 합성 데이터 한계 보완
    \item \textbf{베이스라인 강화}: pyserini BM25, BGE-M3, E5-large 등 공개 모델과의 비교로 일반화 가능성 검증
\end{enumerate}

\subsubsection{중기 과제 (Medium-term)}

\begin{enumerate}
    \setcounter{enumi}{3}
    \item \textbf{PathRAG 고도화}: 현재 BFS 기반 2-hop 탐색에서 원본 PathRAG의 relational path pruning 기법 통합
    \item \textbf{다국어 확장}: 영어-한국어 교차 언어 검색 지원
    \item \textbf{생성 품질 평가}: RAG 전체 파이프라인 (검색 + 생성) 통합 평가, Human evaluation 포함
\end{enumerate}

\subsubsection{장기 과제 (Long-term)}

\begin{enumerate}
    \setcounter{enumi}{6}
    \item \textbf{현장 배포}: 실제 스마트팜 농가에서의 사용자 연구를 통한 실용성 검증
    \item \textbf{연속 학습}: 사용자 피드백 기반 온라인 학습 메커니즘 개발
    \item \textbf{멀티모달 확장}: 잎/줄기 이미지 기반 병해 진단과 RAG 검색 결합, 센서 데이터 통합
\end{enumerate}

% ---

\section*{Acknowledgments}

(감사의 글 작성 예정)
