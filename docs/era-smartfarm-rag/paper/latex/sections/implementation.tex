% Section 4: Implementation
% Converted from: 04_implementation.md

\section{시스템 구현 (Implementation)}
\label{sec:implementation}

\subsection{기술 스택}

\begin{table}[htbp]
	\centering
	\caption{서버 및 엣지 환경별 기술 스택}
	\label{tab:tech_stack}
	\begin{tabular}{@{}llll@{}}
		\toprule
		\textbf{구성요소}    & \textbf{서버 환경}               & \textbf{엣지 환경}       & \textbf{참조}        \\
		\midrule
		Dense Retrieval  & FAISS + Qwen3-Embedding-0.6B & FAISS + MiniLM-L6    & \cite{ref25,ref26} \\
		Sparse Retrieval & TF-IDF (scikit-learn)        & TF-IDF (동일)          & -                  \\
		지식 그래프           & 커스텀 그래프 (JSON)               & 서브셋 그래프              & -                  \\
		LLM              & llama.cpp (FP16/INT8)        & llama.cpp (Q4\_K\_M) & \cite{ref23}       \\
		API              & FastAPI + Docker             & FastAPI (경량)         & -                  \\
		오프라인 폴백          & -                            & 캐시 + 규칙 기반           & \cite{ref24,ref29} \\
		\bottomrule
	\end{tabular}
\end{table}

\subsection{핵심 모듈 구현}

\subsubsection{HybridDATRetriever}

3채널 검색 융합과 후처리를 담당하는 핵심 리트리버이다.

\textbf{주요 파라미터:}
\begin{itemize}
	\item \texttt{DEDUP\_THRESHOLD}: 0.85 (시맨틱 중복 판단 임계값)
	\item \texttt{CROP\_MATCH\_BONUS}: 0.5 (작물 일치 시 스코어 보너스)
	\item \texttt{CROP\_MISMATCH\_PENALTY}: 0.85 (작물 불일치 시 패널티 계수)
\end{itemize}

\textbf{동적 가중치 계산 (\texttt{dynamic\_alphas}):}
온톨로지 매칭 결과와 수치/단위 패턴을 분석하여 3채널 가중치 반환. LLM 호출 없이 규칙 기반으로 동작하여 엣지 환경 최적화.

\subsubsection{PathRAGRetriever (PathRAG-lite)}

PathRAG~\cite{ref8}의 경로 탐색 개념을 차용한 경량 구현이다. 원본 PathRAG의 relational path pruning 대신 BFS(Breadth-First Search) 기반 단순화된 탐색을 수행한다.

\textbf{탐색 전략:}
\begin{itemize}
	\item 시작점: 쿼리에서 매칭된 온톨로지 개념 노드 (예: ``와사비 고수온'' $\rightarrow$ crop:와사비, env:고수온)
	\item 최대 깊이: 2-hop (기본값)
	\item 인과관계 엣지(\texttt{causes}, \texttt{solved\_by}) 우선 탐색
\end{itemize}

\subsubsection{GraphBuilder}

문서 인제스트 시 인과관계 그래프를 자동 구축한다.

\textbf{인과관계 패턴:}
\begin{lstlisting}[basicstyle=\small\ttfamily]
CAUSE_PATTERNS: "원인", "이유", "때문", "~하면", "높으면", "낮으면"
EFFECT_PATTERNS: "결과", "영향", "증상", "문제", "장애", "저하"
SOLUTION_PATTERNS: "해결", "대응", "방법", "조치", "관리", "예방"
\end{lstlisting}

\textbf{엣지 생성 로직:}
\begin{enumerate}
	\item 각 문서의 인과관계 역할 탐지 (cause/effect/solution)
	\item 공통 키워드(작물, 환경요소, 병해, 상태) 추출
	\item 키워드 교집합이 존재하는 문서 쌍에 엣지 생성
\end{enumerate}

\subsubsection{EmbeddingRetriever}

FAISS(Facebook AI Similarity Search) 기반 Dense 검색을 담당한다.

\textbf{특징:}
\begin{itemize}
	\item Lazy loading: 시작 시가 아닌 첫 쿼리 시점에 모델 로드 $\rightarrow$ 초기 메모리 절약
	\item L2 정규화된 임베딩으로 코사인 유사도 검색
	\item mmap 지원으로 대용량 인덱스도 저메모리에서 사용 가능
\end{itemize}

\subsection{메모리 적응형 리랭킹}

런타임 가용 메모리에 따라 리랭커를 동적으로 선택한다:

\begin{table}[htbp]
	\centering
	\caption{메모리 적응형 리랭킹 설정}
	\label{tab:reranking_impl}
	\begin{tabular}{@{}lll@{}}
		\toprule
		\textbf{가용 RAM}    & \textbf{리랭커} & \textbf{설명}         \\
		\midrule
		$<$ 0.8GB          & none         & 리랭킹 비활성화            \\
		0.8GB $\sim$ 1.5GB & LLM-lite     & llama.cpp 기반 경량 리랭킹 \\
		$\geq$ 1.5GB       & BGE          & BGE-reranker-v2-m3  \\
		\bottomrule
	\end{tabular}
\end{table}

\textbf{설정 파라미터:}
\begin{itemize}
	\item \texttt{AUTO\_RERANK\_MIN\_RAM\_GB}: 0.8
	\item \texttt{AUTO\_BGE\_MIN\_RAM\_GB}: 1.5
	\item \texttt{AUTO\_BGE\_MIN\_VRAM\_GB}: 1.5 (GPU 사용 시)
\end{itemize}

\subsection{인덱스 영속화}

오프라인 환경 지원을 위해 인덱스를 파일로 저장/로드한다:

\begin{table}[htbp]
	\centering
	\caption{인덱스 파일 구조}
	\label{tab:index_files}
	\begin{tabular}{@{}lll@{}}
		\toprule
		\textbf{파일}                & \textbf{내용}      & \textbf{형식}      \\
		\midrule
		\texttt{dense.faiss}       & 문서 임베딩 벡터 인덱스    & Binary (mmap 가능) \\
		\texttt{dense\_docs.jsonl} & 문서 텍스트 및 메타데이터   & JSON Lines       \\
		\texttt{sparse.pkl}        & TF-IDF 키워드 빈도 행렬 & Pickle           \\
		\bottomrule
	\end{tabular}
\end{table}

\subsection{그래프 스키마}

CropDP-KG~\cite{ref12}와 AgriKG~\cite{ref21}의 스키마 설계를 참조하여 구성하였다.

\textbf{노드 타입}: practice(문서), crop, env, disease, nutrient, stage

\textbf{엣지 타입:}
\begin{table}[htbp]
	\centering
	\caption{그래프 엣지 타입 정의}
	\label{tab:edge_types}
	\begin{tabular}{@{}lll@{}}
		\toprule
		\textbf{타입}               & \textbf{의미}         & \textbf{참조}              \\
		\midrule
		\texttt{recommended\_for} & 작물 $\rightarrow$ 실천 & AgriKG~\cite{ref21}      \\
		\texttt{associated\_with} & 병해 $\rightarrow$ 실천 & CropDP-KG~\cite{ref12}   \\
		\texttt{mentions}         & 실천 $\rightarrow$ 개념 & 농업 온톨로지~\cite{ref10}     \\
		\texttt{causes}           & 실천 $\rightarrow$ 실천 & 인과 추출~\cite{ref14,ref15} \\
		\texttt{solved\_by}       & 실천 $\rightarrow$ 실천 & 인과 추출~\cite{ref14,ref15} \\
		\bottomrule
	\end{tabular}
\end{table}

\subsection{엣지 배포 사양}

\begin{table}[htbp]
	\centering
	\caption{환경별 배포 사양 및 지원 기능}
	\label{tab:deployment_spec}
	\begin{tabular}{@{}lllp{3.5cm}@{}}
		\toprule
		\textbf{환경} & \textbf{최소 사양} & \textbf{권장 사양}     & \textbf{지원 기능} \\
		\midrule
		서버          & 32GB RAM, GPU  & 64GB RAM, RTX 4090 & 전체 기능          \\
		엣지 게이트웨이    & 8GB RAM, CPU   & 16GB RAM, CPU/NPU  & RAG + Q4 LLM   \\
		저사양 엣지      & 4GB RAM        & 8GB RAM            & 검색 전용          \\
		IoT 노드      & 512MB RAM      & 1GB RAM            & 센서 + 규칙        \\
		\bottomrule
	\end{tabular}
\end{table}

\subsection{EdgeRAG와의 구현 비교}

\begin{table}[htbp]
	\centering
	\caption{EdgeRAG vs 본 시스템 구현 비교}
	\label{tab:edgerag_comparison}
	\begin{tabular}{@{}lll@{}}
		\toprule
		\textbf{구분} & \textbf{EdgeRAG~\cite{ref24}} & \textbf{본 시스템}               \\
		\midrule
		최적화 초점      & 범용 메모리 최적화                    & 도메인 특화 + 엣지 배포               \\
		인덱싱 전략      & 온라인 계층적 인덱싱                   & 오프라인 사전 인덱싱 + mmap           \\
		검색 채널       & 단일 Dense                      & Dense + Sparse + PathRAG 3채널 \\
		그래프 활용      & 없음                            & 인과관계 그래프                     \\
		도메인 지식      & 범용                            & 농업 온톨로지 6개 유형                \\
		메모리 절감      & 계층적 로딩 50\%$\downarrow$       & 양자화 75\%$\downarrow$ + mmap  \\
		오프라인 지원     & 제한적                           & Sparse 검색 + 캐시 폴백            \\
		품질 향상       & 메모리 효율 우선                     & 작물 필터링, 중복 제거                \\
		\bottomrule
	\end{tabular}
\end{table}

\textbf{핵심 차별점:}
\begin{enumerate}
	\item \textbf{도메인 특화}: EdgeRAG가 범용 메모리 최적화에 집중하는 반면, 본 시스템은 농업 온톨로지와 인과관계 그래프를 활용하여 검색 품질 향상
	\item \textbf{멀티 채널}: 수치/단위 정보(EC, pH)의 정확한 매칭을 위한 Sparse 채널 유지
	\item \textbf{메모리 적응형}: 런타임 가용 메모리에 따른 동적 리랭커 선택
\end{enumerate}
