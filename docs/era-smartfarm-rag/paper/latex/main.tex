% IEEE Access Template - Single Column
% Encoding: UTF-8
\documentclass[onecolumn]{IEEEtran}

% ============================================================================
% Packages
% ============================================================================
\usepackage[utf8]{inputenc}
\usepackage[T1]{fontenc}
\usepackage{silence}  % Suppress specific warnings
\WarningFilter{latexfont}{Font shape}  % Suppress font shape warnings
\WarningFilter{latexfont}{Some font}   % Suppress "Some font shapes" warning
\usepackage[hangul]{kotex}  % Korean language support with hangul option
\usepackage{amsmath,amssymb,amsfonts}
\usepackage{algorithmic}
\usepackage{graphicx}
\graphicspath{
	{figures/}
	{figures/architecture/}
	{figures/methodology/}
	{figures/experiments/}
}
\usepackage{textcomp}
\usepackage{xcolor}
\usepackage{booktabs}
\usepackage{multirow}
\usepackage{hyperref}
\usepackage{cleveref}
\usepackage[caption=false]{subfig}  % Use subfig instead of subcaption for IEEEtran
\usepackage{url}

% ============================================================================
% Document Information
% ============================================================================
\title{경량 LLM 및 RAG 기반 와사비 스마트농업 질의응답 시스템 개발: \\
온디바이스 엣지(On-device Edge) 환경을 위한 경량 LLM 및 RAG 기술 연구}

\author{
    \IEEEauthorblockN{First Author\IEEEauthorrefmark{1}, 
    Second Author\IEEEauthorrefmark{2}, and 
    Third Author\IEEEauthorrefmark{1}}
    \IEEEauthorblockA{\IEEEauthorrefmark{1}Department of Computer Science, 
    University Name, City, Country}
    \IEEEauthorblockA{\IEEEauthorrefmark{2}Research Institute, 
    Organization Name, City, Country}
    \IEEEauthorblockA{Email: \{first, third\}@university.edu, second@research.org}
}

% ============================================================================
% Custom Commands
% ============================================================================
\newcommand{\TODO}[1]{\textcolor{red}{[TODO: #1]}}
\newcommand{\TBD}{\textcolor{orange}{TBD}}

% ============================================================================
% Document Begin
% ============================================================================
\begin{document}

\maketitle

% ----------------------------------------------------------------------------
% Abstract
% ----------------------------------------------------------------------------
\begin{abstract}
	\input{sections/abstract}
\end{abstract}

% ----------------------------------------------------------------------------
% Keywords
% ----------------------------------------------------------------------------
\begin{IEEEkeywords}
	스마트농업, 와사비 재배, 검색 증강 생성(RAG), 경량 LLM, 엣지 컴퓨팅,
	온디바이스 추론, 도메인 온톨로지, 질의응답 시스템
\end{IEEEkeywords}

% ----------------------------------------------------------------------------
% Sections
% ----------------------------------------------------------------------------
\input{sections/introduction}
\input{sections/related_work}
% Section 3: Methodology
% Converted from: 03_methodology.md
% Updated: 2026-01-16 - Applied 2-Lane architecture design (Option B)

\section{제안 방법론 (Proposed Methodology)}
\label{sec:methodology}

\subsection{전체 시스템 개요 및 처리 흐름}

본 연구는 스마트팜 현장의 자원 제약 환경(8GB RAM)에서 근거 기반 실시간 응답을 제공하는 온디바이스 RAG 시스템을 제안한다. Figure~\ref{fig:full_flow}는 연구 전체 생애주기와 시스템 아키텍처를 \textbf{3-레인 타임라인 구조}로 제시한다:

\begin{itemize}
	\item \textbf{System Pipeline (상단, 청색)}: 실제 엣지 디바이스에 배포되는 시스템 구성요소 (7단계)
	\item \textbf{Evaluation Protocol (하단, 주황색)}: 시스템 검증을 위한 연구 방법론 (Section~\ref{sec:experiments}에서 상세 기술)
\end{itemize}

\subsubsection{End-to-End 처리 흐름}

% [Figure 1 placeholder - 3-Lane Timeline Architecture]
\begin{figure*}[htbp]
	\centering
	\fbox{\parbox{0.95\textwidth}{\centering \textbf{[Figure 1: Research Lifecycle \& System Architecture (3-Lane Timeline)]} \\[1em]
			\textcolor{blue}{\textbf{🔵 SYSTEM PIPELINE (Top Lane)}} \\[0.5em]
			\textbf{PHASE 1: Data Preparation (Offline)} \\
			① Data Collection $\rightarrow$ ② Preprocessing (OCR, Chunking) $\rightarrow$ ③ Knowledge Store \\[0.5em]
			$\downarrow$ \textit{mmap load} \\[0.5em]
			\textbf{PHASE 2: Online Inference (Per-query)} \\
			④ Query Analysis $\rightarrow$ ⑤ HybridDAT Retrieval $\rightarrow$ ⑥ Context Shaping \\
			$\rightarrow$ ⑦ LLM Generation $\rightarrow$ OUTPUT \{answer, sources, confidence\} \\[1em]
			\hrulefill \\[0.5em]
			\textcolor{orange}{\textbf{🟠 EVALUATION PROTOCOL (Bottom Lane)}} \\[0.5em]
			Verification | Ablation Study | Benchmark Comparison \\
			$\rightarrow$ Evaluation Metrics (Recall@k, Latency, Memory, LLM-Judge)
		}}
	\caption{제안 시스템의 3-레인 타임라인 아키텍처. System Pipeline(상단)은 실제 배포되는 구성요소를, Evaluation Protocol(하단)은 연구 방법론을 보여준다.}
	\label{fig:full_flow}
\end{figure*}

\textbf{System Pipeline (7단계)}:

\begin{table}[htbp]
	\centering
	\small
	\begin{tabular}{c|l|p{9cm}}
		\hline
		\textbf{단계} & \textbf{구성요소}   & \textbf{설명}                                                                                                                       \\
		\hline
		①           & Data Collection & PDF, 이미지, 텍스트 형태의 농업 매뉴얼/가이드 수집                                                                                                   \\
		②           & Preprocessing   & Text extraction, OCR fallback (auto backend), Sentence-window chunking (CHUNK\_SIZE/STRIDE), Metadata tag (crop, causal, numeric) \\
		③           & Knowledge Store & Dense/Sparse 인덱스 (mmap/TF-IDF), Causal Graph (in-memory, built from docs), Ontology                                               \\
		④           & Query Analysis  & 온톨로지 매칭, Dynamic Alpha (rule-based heuristics)                                                                                    \\
		⑤           & HybridDAT       & 3채널 하이브리드 검색 (Dense FAISS + Sparse TF-IDF + PathRAG BFS 2-hop)                                                                    \\
		⑥           & Context Shaping & 작물 필터링 (+0.5/×0.15), 중복 제거 (θ=0.85), 리랭킹                                                                                          \\
		⑦           & LLM Generation  & llama.cpp Q4\_K\_M 기반 응답 생성, Fallback chain                                                                                       \\
		\hline
	\end{tabular}
\end{table}

\textbf{핵심 설계 원칙}:
\begin{enumerate}
	\item \textbf{오프라인 사전 구축}: 인덱싱/인과관계 그래프(in-memory 빌드)는 1회 오프라인으로 수행하여 런타임 부하 최소화
	\item \textbf{메모리 효율}: mmap 기반 lazy loading으로 전체 인덱스를 RAM에 올리지 않음
	\item \textbf{도메인 특화}: 온톨로지 +Dynamic Alpha 휴리스틱으로 범용 RAG 대비 검색 품질 향상
	\item \textbf{검증 분리}: Groundedness Checks(keyword/source-hit) + Prompt Constraints는 Evaluation Protocol로 분리하여 학술적 규약 준수
\end{enumerate}

\textit{Note: Evaluation Protocol (Verification, Ablation, Benchmark)은 시스템 구성요소가 아닌 연구 방법론으로, Section~\ref{sec:experiments}에서 상세히 다룬다. Verification은 프롬프트 제약 및 키워드 기반 근거 확인으로 구현되며, Benchmark는 내부 베이스라인(Dense-only, Sparse-only, Naive Hybrid)과 비교한다.}

\subsubsection{리소스 제약 및 설계 목표}

엣지 환경의 리소스 제약을 명확히 정의하고, 이를 기반으로 각 컴포넌트를 설계하였다.

% Table: 엣지 환경 리소스 제약 및 설계 목표
% Label: tab:resource_constraints
\begin{table}[htbp]
	\centering
	\caption{엣지 환경 리소스 제약 및 설계 목표}
	\label{tab:resource_constraints}
	\begin{tabular}{@{}lllp{4cm}@{}}
		\toprule
		\textbf{리소스 항목} & \textbf{최소 사양} & \textbf{권장 사양} & \textbf{설계 근거}      \\
		\midrule
		RAM             & 8GB            & 16GB           & Jetson Orin Nano 타겟 \\
		저장공간            & 10GB           & 20GB           & GGUF 모델 + FAISS 인덱스 \\
		목표 지연           & p95 < 500ms    & p95 < 300ms    & 실시간 현장 응답           \\
		LLM 메모리         & $\sim$2.5GB    & $\sim$4GB      & Q4\_K\_M 양자화 기준     \\
		처리량             & 3 QPS          & 8 QPS          & CPU 단독 환경           \\
		\bottomrule
	\end{tabular}
\end{table}


\subsubsection{6계층 아키텍처}

% [Figure 2 placeholder - 6-layer architecture diagram]
\begin{figure}[htbp]
	\centering
	\fbox{\parbox{0.9\textwidth}{\centering \textbf{[Figure 2: 6-Layer Stack Architecture]} \\[1em]
			L5: Application \& Policy (Streamlit UI, FastAPI REST, Offline Fallback) \\
			L4: Generation \& Grounding (Prompt Template, TemplateResponder) \\
			L3: Context Shaping (Crop Filter, Semantic Dedup, Reranking) \\
			L2: Retrieval Core (Dense/Sparse/PathRAG 3-channel) \\
			L1: On-device Knowledge Store (FAISS, Causal Graph, Ontology) \\
			L0: Device \& Runtime (llama.cpp, Embedding, FAISS mmap)
		}}
	\caption{제안 시스템의 6계층 스택 아키텍처}
	\label{fig:architecture}
\end{figure}

\textbf{계층별 핵심 역할}:

% Table: 계층별 역할 및 핵심 컴포넌트
% Label: tab:layers
\begin{table}[htbp]
	\centering
	\caption{계층별 역할 및 핵심 컴포넌트}
	\label{tab:layers}
	\begin{tabular}{@{}clp{5cm}@{}}
		\toprule
		\textbf{계층} & \textbf{역할}     & \textbf{핵심 컴포넌트}          \\
		\midrule
		L5          & 사용자 인터페이스 및 정책  & FastAPI, Streamlit, 폴백 정책 \\
		L4          & 응답 생성 및 그라운딩    & 프롬프트 템플릿, 템플릿 응답기         \\
		L3          & 컨텍스트 압축 (논문 핵심) & 작물 필터, 중복 제거, 리랭킹         \\
		L2          & 3채널 하이브리드 검색    & Dense, Sparse, PathRAG 융합 \\
		L1          & 온디바이스 지식 저장소    & FAISS 인덱스, 인과관계 그래프, 온톨로지 \\
		L0          & 디바이스 런타임        & llama.cpp, 임베딩 모델, FAISS  \\
		\bottomrule
	\end{tabular}
\end{table}


%----------------------------------------------------------------------
\subsection{데이터 수집 및 전처리 파이프라인}

\subsubsection{데이터 수집}

본 연구의 지식 베이스는 세 가지 유형의 농업 문서로 구성된다: (1) 재배 매뉴얼 (농촌진흥청, 도농업기술원 PDF), (2) 기술 가이드 (스마트팜 코리아, 농업기술실용화재단 웹/PDF), (3) 작업 기록 (현장 농가 메모, Q\&A 게시판).

\subsubsection{전처리 파이프라인}

% [Figure 3 placeholder - Preprocessing Pipeline]
\begin{figure}[htbp]
	\centering
	\fbox{\parbox{0.9\textwidth}{\centering \textbf{[Figure 3: Data Preprocessing Pipeline]} \\[1em]
			입력 (PDF/이미지/텍스트) $\rightarrow$ OCR 처리 (EasyOCR) \\
			$\rightarrow$ 시맨틱 청킹 (200-500 토큰) $\rightarrow$ 메타데이터 추출 \\
			(crop, category, causal\_role, numeric\_info, source)
		}}
	\caption{데이터 전처리 파이프라인}
	\label{fig:preprocessing}
\end{figure}

\subsubsection{OCR 및 텍스트 정규화}

이미지 기반 문서는 EasyOCR을 사용하여 텍스트로 변환한다. 정규화 규칙을 적용하여 온도(25도 $\rightarrow$ 25℃), EC(전기전도도 2.5 $\rightarrow$ EC 2.5 dS/m), pH(산도 6.5 $\rightarrow$ pH 6.5) 등 수치 표현을 통일한다.

\subsubsection{시맨틱 청킹 전략}

단순 길이 기반 분할 대신, 문서의 의미 구조를 보존하는 청킹을 적용한다. 섹션 기반 1차 분할 후 짧은 섹션은 연관 섹션과 병합하며, 50 토큰 오버랩으로 문맥 연속성을 확보한다. 청킹 파라미터: \texttt{CHUNK\_MIN\_TOKENS}=200, \texttt{CHUNK\_MAX\_TOKENS}=500, \texttt{CHUNK\_OVERLAP}=50.

\subsubsection{메타데이터 자동 추출}

각 청크에 대해 crop(작물), category(카테고리), causal\_role(인과관계 역할), numeric\_info(수치 정보), source(출처) 메타데이터를 규칙 기반으로 자동 추출한다.

%----------------------------------------------------------------------
\subsection{스마트팜 온톨로지}

\subsubsection{설계 배경}

온톨로지 설계는 Stanford 온톨로지 구축 방법론~\cite{ref13}과 기존 농업 온톨로지 연구~\cite{ref9,ref10,ref11}를 참조하여 스마트팜 도메인에 적합한 6개 개념 유형을 정의하였다. CropDP-KG~\cite{ref12}의 엔티티 구조와 AgriKG~\cite{ref21}의 농업 엔티티 분류를 참고하여 한국 스마트팜 환경에 맞게 구성하였다.

\subsubsection{개념 유형 정의}

% Table: 스마트팜 도메인 온톨로지 개념 유형
% Label: tab:ontology
\begin{table}[htbp]
	\centering
	\caption{스마트팜 도메인 온톨로지 개념 유형}
	\label{tab:ontology}
	\begin{tabular}{@{}llll@{}}
		\toprule
		\textbf{유형} & \textbf{설명} & \textbf{예시}            & \textbf{근거}                  \\
		\midrule
		crop        & 재배 작물       & 와사비, 토마토, 딸기           & CropDP-KG~\cite{ref12}       \\
		env         & 환경 요소       & 온도, 습도, EC, pH, CO$_2$ & IoT 센서 표준~\cite{ref11}       \\
		nutrient    & 영양소         & 양액, 비료, 관수             & 농업 지식 베이스~\cite{ref9}        \\
		disease     & 병해충         & 흰가루병, 뿌리썩음병, 연부병       & CropDP-KG~\cite{ref12,ref13} \\
		stage       & 생육 단계       & 육묘, 정식, 생육, 수확         & 작물 생육 모델~\cite{ref17}        \\
		practice    & 재배 실천       & 차광, 환기, 난방, 살균         & 농업 실천 온톨로지~\cite{ref9,ref10} \\
		\bottomrule
	\end{tabular}
\end{table}


각 개념은 동의어/유의어 목록(alias)을 포함한다. 예를 들어 ``와사비''의 alias에는 ``산와사비'', ``본와사비''가 포함되어 사용자가 어떤 표현을 쓰더라도 동일 개념으로 인식한다.

%----------------------------------------------------------------------
\subsection{3채널 하이브리드 검색 (HybridDAT)}

\subsubsection{설계 근거}

Dense retrieval은 의미적 유사성 검색에 강하지만 ``EC 2.5 dS/m'' 같은 수치 정보 매칭에 취약하다. Sparse retrieval은 정확한 키워드 매칭에 강하지만 의미적 유사성을 놓칠 수 있다~\cite{ref5}. 본 시스템은 Dense-Sparse-PathRAG 3채널 융합과 질의 특성에 따른 동적 가중치 조정(Dynamic Alpha Tuning)을 적용한다.

\subsubsection{HybridDATRetriever 플로우}

% [Figure 4 placeholder - HybridDAT flow diagram]
\begin{figure}[htbp]
	\centering
	\fbox{\parbox{0.9\textwidth}{\centering \textbf{[Figure 4: HybridDAT Retrieval Flow]} \\[1em]
			1. Query $\rightarrow$ Ontology Matching \\
			2. Dynamic Alpha Calculation (수치/병해 분석) \\
			3. 3-Channel Parallel Search (Dense/Sparse/PathRAG) \\
			4. Score Fusion (Min-Max Normalization + Weighted Sum) \\
			5. Top-k $\times$ 2 Candidates
		}}
	\caption{HybridDATRetriever 검색 플로우}
	\label{fig:hybriddat}
\end{figure}

\subsubsection{동적 가중치 규칙 (Dynamic Alpha)}

질의 내용을 분석하여 가중치를 자동 결정한다:

% Table: 질의 특성에 따른 동적 가중치 규칙
% Label: tab:dynamic_alpha
\begin{table}[htbp]
	\centering
	\caption{질의 특성에 따른 동적 가중치 규칙}
	\label{tab:dynamic_alpha}
	\begin{tabular}{@{}lcccl@{}}
		\toprule
		\textbf{질의 특성} & $\alpha_d$ & $\alpha_s$ & $\alpha_p$ & \textbf{설계 근거}          \\
		\midrule
		일반 질의          & 0.5        & 0.5        & 0.0        & 의미 검색과 키워드 매칭 균형        \\
		수치/단위 포함       & 0.3        & 0.7        & 0.0        & 수치 정확 매칭 필요~\cite{ref4} \\
		병해/재배 관련       & 0.35       & 0.35       & 0.3        & 인과관계 탐색 활성화             \\
		\bottomrule
	\end{tabular}
\end{table}


%----------------------------------------------------------------------
\subsection{인과관계 그래프 (PathRAG-lite)}

\subsubsection{설계 배경}

농업 도메인에서 ``고수온 $\rightarrow$ 연부병 발생 $\rightarrow$ 수온 관리'' 같은 인과 체인이 핵심 정보 구조를 형성한다~\cite{ref17}. GraphRAG~\cite{ref7}는 LLM으로 개체와 관계를 추출하므로 구축 비용이 높다(문서 1000개당 GPT-4 \$100+). 본 시스템은 규칙 기반 패턴 매칭으로 인과관계 그래프를 구축하여 비용을 \$0으로 절감한다.

\subsubsection{인과관계 역할 분류}

텍스트 패턴 매칭으로 문서의 역할을 분류한다:

% Table: 인과관계 역할 분류 패턴
% Label: tab:causal_patterns
\begin{table}[htbp]
	\centering
	\caption{인과관계 역할 분류 패턴}
	\label{tab:causal_patterns}
	\begin{tabular}{@{}lll@{}}
		\toprule
		\textbf{역할} & \textbf{판별 패턴}               & \textbf{예시 문장}                     \\
		\midrule
		Cause       & ``원인'', ``때문'', ``$\sim$하면'' & ``고온 환경에서는 화분 활력이 저하된다''           \\
		Effect      & ``결과'', ``증상'', ``문제''       & ``착과율이 떨어지는 문제가 발생한다''             \\
		Solution    & ``관리'', ``해야'', ``방법''       & ``야간 온도를 18$^\circ$C 이하로 관리해야 한다'' \\
		\bottomrule
	\end{tabular}
\end{table}


\subsubsection{PathRAG-lite BFS 탐색}

PathRAG~\cite{ref8}의 경로 탐색 개념을 차용한 경량 구현이다. BFS(너비 우선 탐색) 기반 2-hop 탐색으로 원인$\rightarrow$결과$\rightarrow$해결책 문서를 수집한다.

\subsubsection{그래프 스키마}

CropDP-KG~\cite{ref12}와 AgriKG~\cite{ref21}의 스키마 설계를 참조하여 구성하였다.

\textbf{노드 타입}: practice(문서), crop, env, disease, nutrient, stage

\textbf{엣지 타입}:
\begin{itemize}
	\item \texttt{recommended\_for}: 작물 $\rightarrow$ 실천 (AgriKG~\cite{ref21})
	\item \texttt{associated\_with}: 병해 $\rightarrow$ 실천 (CropDP-KG~\cite{ref12})
	\item \texttt{mentions}: 실천 $\rightarrow$ 개념 (농업 온톨로지~\cite{ref10})
	\item \texttt{causes}: 실천 $\rightarrow$ 실천 (인과 추출~\cite{ref14,ref15})
	\item \texttt{solved\_by}: 실천 $\rightarrow$ 실천 (인과 추출~\cite{ref14,ref15})
\end{itemize}

%----------------------------------------------------------------------
\subsection{Context Shaping (컨텍스트 압축)}

엣지 LLM은 토큰이 곧 지연/전력 비용이므로, 검색 결과를 그대로 전달하지 않고 압축/필터링하는 것이 핵심이다.

\subsubsection{Context Shaping 파이프라인}

% [Figure 5 placeholder - Context Shaping pipeline]
\begin{figure}[htbp]
	\centering
	\fbox{\parbox{0.9\textwidth}{\centering \textbf{[Figure 5: Context Shaping Pipeline]} \\[1em]
			Input (16 docs) $\rightarrow$ Crop Filter ($\sim$12 docs) \\
			$\rightarrow$ Semantic Dedup ($\sim$8 docs) $\rightarrow$ Reranking $\rightarrow$ Output (4 docs)
		}}
	\caption{Context Shaping 파이프라인}
	\label{fig:context_shaping}
\end{figure}

\subsubsection{작물 필터링 (Crop-aware Filtering)}

농업 지식 그래프 연구~\cite{ref4,ref12}에서 작물별 맥락 의존성이 강조되었다. 질의의 작물과 문서의 작물 메타데이터를 비교하여 스코어를 조정한다.

% Table: 작물 필터링 스코어 조정
% Label: tab:crop_filter
\begin{table}[htbp]
	\centering
	\caption{작물 필터링 스코어 조정}
	\label{tab:crop_filter}
	\begin{tabular}{@{}lll@{}}
		\toprule
		\textbf{조건} & \textbf{스코어 조정} & \textbf{효과}  \\
		\midrule
		작물 일치       & +0.5            & 관련 문서 우선     \\
		작물 불일치      & $\times$0.15    & 무관한 작물 정보 억제 \\
		작물 정보 없음    & 유지              & 일반 정보 보존     \\
		\bottomrule
	\end{tabular}
\end{table}


\subsubsection{시맨틱 중복 제거 (Semantic Deduplication)}

MMR~\cite{ref18}과 VRSD~\cite{ref19}를 참조하여 검색 결과의 다양성을 확보한다. 두 문서의 임베딩 벡터 간 코사인 유사도가 임계값($\theta$=0.85) 이상인 문서 쌍에서 후순위 문서를 제거한다.

\subsubsection{메모리 적응형 리랭킹}

런타임 가용 메모리에 따라 리랭커를 동적으로 선택한다:

% Table: 메모리 적응형 리랭킹 전략
% Label: tab:reranking
\begin{table}[htbp]
	\centering
	\caption{메모리 적응형 리랭킹 전략}
	\label{tab:reranking}
	\begin{tabular}{@{}llll@{}}
		\toprule
		\textbf{가용 RAM}    & \textbf{리랭커} & \textbf{추가 메모리} & \textbf{설명}        \\
		\midrule
		$<$ 0.8GB          & none         & 0MB             & 리랭킹 비활성화           \\
		0.8GB $\sim$ 1.5GB & LLM-lite     & $\sim$0MB       & llama.cpp 재사용      \\
		$\geq$ 1.5GB       & BGE          & $\sim$500MB     & BGE-reranker-v2-m3 \\
		\bottomrule
	\end{tabular}
\end{table}


%----------------------------------------------------------------------
\subsection{엣지 배포 최적화}

\subsubsection{메모리 계층 구조 (RAM vs Flash)}

엣지 환경에서 ``벡터 인덱스가 RAM에 다 못 올라간다''는 병목을 해결하기 위해 계층적 메모리 구조를 설계하였다.

\begin{itemize}
	\item \textbf{RAM (Hot Data)}: Query Cache (LRU 128), Embedding Cache (LRU 256), FAISS mmap Active Pages, LLM Weights ($\sim$2.5GB)
	\item \textbf{Flash/SSD (Cold Data)}: dense.faiss (mmap), sparse.pkl, responses.jsonl, Causal Graph (in-memory built)
\end{itemize}

\subsubsection{LLM 양자화 전략}

llama.cpp의 GGUF 포맷~\cite{ref23}을 활용하여 Q4\_K\_M 양자화를 기본으로 적용한다.

% Table: LLM 양자화 수준별 비교
% Label: tab:quantization
\begin{table}[htbp]
	\centering
	\caption{LLM 양자화 수준별 비교 (4B 모델 기준)}
	\label{tab:quantization}
	\begin{tabular}{@{}llll@{}}
		\toprule
		\textbf{양자화 수준}   & \textbf{메모리} & \textbf{품질 손실} & \textbf{적용 환경}      \\
		\midrule
		FP16 (원본)         & $\sim$8GB    & 없음             & 서버 환경 (GPU 필수)      \\
		INT8              & $\sim$4GB    & 최소             & 고사양 엣지 (8GB RAM)    \\
		\textbf{Q4\_K\_M} & $\sim$2.5GB  & 낮음             & \textbf{일반 엣지 (권장)} \\
		Q2\_K             & $\sim$1.5GB  & 중간             & 극저사양 환경             \\
		\bottomrule
	\end{tabular}
\end{table}


Q4\_K\_M은 중요한 레이어는 5비트, 나머지는 4비트로 혼합 양자화하여 품질 대비 메모리 효율의 최적점으로 평가된다.

\subsubsection{오프라인 폴백 모드}

네트워크 단절 또는 LLM 장애 시 다음과 같은 폴백 전략을 적용한다:

\begin{enumerate}
	\item \textbf{Similar Cache}: ResponseCache.get\_similar() - 임베딩 유사도 $\geq$ 0.9인 이전 유사 질의 응답 재활용
	\item \textbf{Template Response}: TemplateResponder.generate() - 온톨로지 매칭 기반 정형화된 응답 생성
	\item \textbf{Search Only}: LLM 없이 검색 결과만 반환
\end{enumerate}

%----------------------------------------------------------------------
\subsection{런타임 검증 및 신뢰도 표시}

\textit{Note: 본 섹션은 시스템에 내장된 런타임 검증 기능을 다룬다. Ablation Study, Benchmark 비교 등 연구 방법론으로서의 평가는 Section~\ref{sec:experiments}에서 상세히 기술한다.}

엣지 환경에서 LLM의 환각(hallucination) 위험을 완화하기 위해, 시스템은 응답 생성 시점에 다음과 같은 런타임 검증 메커니즘을 수행한다.

\subsubsection{근거 추적 (Source Attribution)}

생성된 응답의 각 주장에 대해 근거 문서를 명시적으로 연결한다. LLM 프롬프트에 검색된 문서와 함께 ``근거를 명시하라''는 지시를 포함하고, 응답 생성 후 주장-문서 간 임베딩 유사도를 계산하여 유사도가 임계값(0.7) 미만인 주장에 대해 경고를 표시한다.

\subsubsection{런타임 환각 감지}

% [Figure 6 placeholder - Hallucination Detection]
\begin{figure}[htbp]
	\centering
	\fbox{\parbox{0.9\textwidth}{\centering \textbf{[Figure 6: Verification Pipeline]} \\[1em]
			1. Claim Extraction (응답에서 사실적 주장 추출) \\
			2. Evidence Matching (주장-문서 유사도 계산) \\
			3. Consistency Check (수치 정보 일치 확인) \\
			4. Confidence Scoring (HIGH/MEDIUM/LOW)
		}}
	\caption{응답 검증 파이프라인}
	\label{fig:verification}
\end{figure}

\subsubsection{수치 정보 검증}

농업 도메인에서 수치 정보의 정확성은 특히 중요하다. 범위 검증(수온 10-25℃, pH 5.5-7.5 등 도메인 지식 기반 허용 범위), 일관성 검증(근거 문서 내 수치와 비교), 단위 검증(단위 변환 정확성 확인)을 적용한다.

\subsubsection{신뢰도 표시}

최종 응답에는 신뢰도 수준이 함께 제공된다:
\begin{itemize}
	\item \textbf{HIGH}: 모든 주장에 유사도 $\geq$0.8 근거 존재 - 응답 신뢰 가능
	\item \textbf{MEDIUM}: 일부 주장만 근거 확인 ($\geq$60\%) - 추가 확인 권장
	\item \textbf{LOW}: 근거 확인 불가 ($<$60\%) - 전문가 상담 권장
\end{itemize}

%----------------------------------------------------------------------
\subsection{관련 연구와의 차별점}

\textit{Note: 관련 연구에 대한 포괄적인 리뷰는 Section~\ref{sec:related_work}를 참조한다. 본 섹션에서는 제안 방법론의 핵심 차별점을 요약한다.}

% Table: EdgeRAG vs ERA-SmartFarm-RAG 비교
% Label: tab:comparison
\begin{table}[htbp]
	\centering
	\caption{EdgeRAG vs ERA-SmartFarm-RAG 비교}
	\label{tab:comparison}
	\begin{tabular}{@{}lll@{}}
		\toprule
		\textbf{구분} & \textbf{EdgeRAG~\cite{ref24}} & \textbf{ERA-SmartFarm-RAG}  \\
		\midrule
		최적화 초점      & 범용 메모리 최적화                    & 도메인 특화 + 엣지 배포              \\
		인덱싱 전략      & 온라인 계층적 인덱싱                   & 오프라인 사전 인덱싱 + mmap          \\
		검색 채널       & 단일 Dense                      & Dense + Sparse + PathRAG    \\
		그래프 활용      & 없음                            & 인과관계 그래프                    \\
		도메인 지식      & 범용                            & 농업 온톨로지 6개 유형               \\
		메모리 절감      & 계층적 로딩 50\%$\downarrow$       & 양자화 75\%$\downarrow$ + mmap \\
		오프라인 지원     & 제한적                           & 폴백 체인                       \\
		\bottomrule
	\end{tabular}
\end{table}


% Section 4: Implementation
% Converted from: 04_implementation.md

\section{시스템 구현 (Implementation)}
\label{sec:implementation}

\subsection{기술 스택}

\begin{table}[htbp]
	\centering
	\caption{서버 및 엣지 환경별 기술 스택}
	\label{tab:tech_stack}
	\begin{tabular}{@{}llll@{}}
		\toprule
		\textbf{구성요소}    & \textbf{서버 환경}               & \textbf{엣지 환경}       & \textbf{참조}        \\
		\midrule
		Dense Retrieval  & FAISS + Qwen3-Embedding-0.6B & FAISS + MiniLM-L6    & \cite{ref25,ref26} \\
		Sparse Retrieval & TF-IDF (scikit-learn)        & TF-IDF (동일)          & -                  \\
		지식 그래프           & 커스텀 그래프 (JSON)               & 서브셋 그래프              & -                  \\
		LLM              & llama.cpp (FP16/INT8)        & llama.cpp (Q4\_K\_M) & \cite{ref23}       \\
		API              & FastAPI + Docker             & FastAPI (경량)         & -                  \\
		오프라인 폴백          & -                            & 캐시 + 규칙 기반           & \cite{ref24,ref29} \\
		\bottomrule
	\end{tabular}
\end{table}

\subsection{핵심 모듈 구현}

\subsubsection{HybridDATRetriever}

3채널 검색 융합과 후처리를 담당하는 핵심 리트리버이다.

\textbf{주요 파라미터:}
\begin{itemize}
	\item \texttt{DEDUP\_THRESHOLD}: 0.85 (시맨틱 중복 판단 임계값)
	\item \texttt{CROP\_MATCH\_BONUS}: 0.5 (작물 일치 시 스코어 보너스)
	\item \texttt{CROP\_MISMATCH\_PENALTY}: 0.85 (작물 불일치 시 패널티 계수)
\end{itemize}

\textbf{동적 가중치 계산 (\texttt{dynamic\_alphas}):}
온톨로지 매칭 결과와 수치/단위 패턴을 분석하여 3채널 가중치 반환. LLM 호출 없이 규칙 기반으로 동작하여 엣지 환경 최적화.

\subsubsection{PathRAGRetriever (PathRAG-lite)}

PathRAG~\cite{ref8}의 경로 탐색 개념을 차용한 경량 구현이다. 원본 PathRAG의 relational path pruning 대신 BFS(Breadth-First Search) 기반 단순화된 탐색을 수행한다.

\textbf{탐색 전략:}
\begin{itemize}
	\item 시작점: 쿼리에서 매칭된 온톨로지 개념 노드 (예: ``와사비 고수온'' $\rightarrow$ crop:와사비, env:고수온)
	\item 최대 깊이: 2-hop (기본값)
	\item 인과관계 엣지(\texttt{causes}, \texttt{solved\_by}) 우선 탐색
\end{itemize}

\subsubsection{GraphBuilder}

문서 인제스트 시 인과관계 그래프를 자동 구축한다.

\textbf{인과관계 패턴:}
\begin{lstlisting}[basicstyle=\small\ttfamily]
CAUSE_PATTERNS: "원인", "이유", "때문", "~하면", "높으면", "낮으면"
EFFECT_PATTERNS: "결과", "영향", "증상", "문제", "장애", "저하"
SOLUTION_PATTERNS: "해결", "대응", "방법", "조치", "관리", "예방"
\end{lstlisting}

\textbf{엣지 생성 로직:}
\begin{enumerate}
	\item 각 문서의 인과관계 역할 탐지 (cause/effect/solution)
	\item 공통 키워드(작물, 환경요소, 병해, 상태) 추출
	\item 키워드 교집합이 존재하는 문서 쌍에 엣지 생성
\end{enumerate}

\subsubsection{EmbeddingRetriever}

FAISS(Facebook AI Similarity Search) 기반 Dense 검색을 담당한다.

\textbf{특징:}
\begin{itemize}
	\item Lazy loading: 시작 시가 아닌 첫 쿼리 시점에 모델 로드 $\rightarrow$ 초기 메모리 절약
	\item L2 정규화된 임베딩으로 코사인 유사도 검색
	\item mmap 지원으로 대용량 인덱스도 저메모리에서 사용 가능
\end{itemize}

\subsection{메모리 적응형 리랭킹}

런타임 가용 메모리에 따라 리랭커를 동적으로 선택한다:

\begin{table}[htbp]
	\centering
	\caption{메모리 적응형 리랭킹 설정}
	\label{tab:reranking_impl}
	\begin{tabular}{@{}lll@{}}
		\toprule
		\textbf{가용 RAM}    & \textbf{리랭커} & \textbf{설명}         \\
		\midrule
		$<$ 0.8GB          & none         & 리랭킹 비활성화            \\
		0.8GB $\sim$ 1.5GB & LLM-lite     & llama.cpp 기반 경량 리랭킹 \\
		$\geq$ 1.5GB       & BGE          & BGE-reranker-v2-m3  \\
		\bottomrule
	\end{tabular}
\end{table}

\textbf{설정 파라미터:}
\begin{itemize}
	\item \texttt{AUTO\_RERANK\_MIN\_RAM\_GB}: 0.8
	\item \texttt{AUTO\_BGE\_MIN\_RAM\_GB}: 1.5
	\item \texttt{AUTO\_BGE\_MIN\_VRAM\_GB}: 1.5 (GPU 사용 시)
\end{itemize}

\subsection{인덱스 영속화}

오프라인 환경 지원을 위해 인덱스를 파일로 저장/로드한다:

\begin{table}[htbp]
	\centering
	\caption{인덱스 파일 구조}
	\label{tab:index_files}
	\begin{tabular}{@{}lll@{}}
		\toprule
		\textbf{파일}                & \textbf{내용}      & \textbf{형식}      \\
		\midrule
		\texttt{dense.faiss}       & 문서 임베딩 벡터 인덱스    & Binary (mmap 가능) \\
		\texttt{dense\_docs.jsonl} & 문서 텍스트 및 메타데이터   & JSON Lines       \\
		\texttt{sparse.pkl}        & TF-IDF 키워드 빈도 행렬 & Pickle           \\
		\bottomrule
	\end{tabular}
\end{table}

\subsection{그래프 스키마}

CropDP-KG~\cite{ref12}와 AgriKG~\cite{ref21}의 스키마 설계를 참조하여 구성하였다.

\textbf{노드 타입}: practice(문서), crop, env, disease, nutrient, stage

\textbf{엣지 타입:}
\begin{table}[htbp]
	\centering
	\caption{그래프 엣지 타입 정의}
	\label{tab:edge_types}
	\begin{tabular}{@{}lll@{}}
		\toprule
		\textbf{타입}               & \textbf{의미}         & \textbf{참조}              \\
		\midrule
		\texttt{recommended\_for} & 작물 $\rightarrow$ 실천 & AgriKG~\cite{ref21}      \\
		\texttt{associated\_with} & 병해 $\rightarrow$ 실천 & CropDP-KG~\cite{ref12}   \\
		\texttt{mentions}         & 실천 $\rightarrow$ 개념 & 농업 온톨로지~\cite{ref10}     \\
		\texttt{causes}           & 실천 $\rightarrow$ 실천 & 인과 추출~\cite{ref14,ref15} \\
		\texttt{solved\_by}       & 실천 $\rightarrow$ 실천 & 인과 추출~\cite{ref14,ref15} \\
		\bottomrule
	\end{tabular}
\end{table}

\subsection{엣지 배포 사양}

\begin{table}[htbp]
	\centering
	\caption{환경별 배포 사양 및 지원 기능}
	\label{tab:deployment_spec}
	\begin{tabular}{@{}lllp{3.5cm}@{}}
		\toprule
		\textbf{환경} & \textbf{최소 사양} & \textbf{권장 사양}     & \textbf{지원 기능} \\
		\midrule
		서버          & 32GB RAM, GPU  & 64GB RAM, RTX 4090 & 전체 기능          \\
		엣지 게이트웨이    & 8GB RAM, CPU   & 16GB RAM, CPU/NPU  & RAG + Q4 LLM   \\
		저사양 엣지      & 4GB RAM        & 8GB RAM            & 검색 전용          \\
		IoT 노드      & 512MB RAM      & 1GB RAM            & 센서 + 규칙        \\
		\bottomrule
	\end{tabular}
\end{table}

\subsection{EdgeRAG와의 구현 비교}

\begin{table}[htbp]
	\centering
	\caption{EdgeRAG vs 본 시스템 구현 비교}
	\label{tab:edgerag_comparison}
	\begin{tabular}{@{}lll@{}}
		\toprule
		\textbf{구분} & \textbf{EdgeRAG~\cite{ref24}} & \textbf{본 시스템}               \\
		\midrule
		최적화 초점      & 범용 메모리 최적화                    & 도메인 특화 + 엣지 배포               \\
		인덱싱 전략      & 온라인 계층적 인덱싱                   & 오프라인 사전 인덱싱 + mmap           \\
		검색 채널       & 단일 Dense                      & Dense + Sparse + PathRAG 3채널 \\
		그래프 활용      & 없음                            & 인과관계 그래프                     \\
		도메인 지식      & 범용                            & 농업 온톨로지 6개 유형                \\
		메모리 절감      & 계층적 로딩 50\%$\downarrow$       & 양자화 75\%$\downarrow$ + mmap  \\
		오프라인 지원     & 제한적                           & Sparse 검색 + 캐시 폴백            \\
		품질 향상       & 메모리 효율 우선                     & 작물 필터링, 중복 제거                \\
		\bottomrule
	\end{tabular}
\end{table}

\textbf{핵심 차별점:}
\begin{enumerate}
	\item \textbf{도메인 특화}: EdgeRAG가 범용 메모리 최적화에 집중하는 반면, 본 시스템은 농업 온톨로지와 인과관계 그래프를 활용하여 검색 품질 향상
	\item \textbf{멀티 채널}: 수치/단위 정보(EC, pH)의 정확한 매칭을 위한 Sparse 채널 유지
	\item \textbf{메모리 적응형}: 런타임 가용 메모리에 따른 동적 리랭커 선택
\end{enumerate}

\input{sections/experiments}
\input{sections/conclusion}

% ----------------------------------------------------------------------------
% References
% ----------------------------------------------------------------------------
\bibliographystyle{IEEEtran}
\bibliography{references}

% ----------------------------------------------------------------------------
% Appendices (Optional)
% ----------------------------------------------------------------------------
% \appendices
% \input{sections/appendix}

\end{document}
